\IfFileExists{stacks-project.cls}{%
\documentclass{stacks-project}
}{%
\documentclass{amsart}
}

% The following AMS packages are automatically loaded with
% the amsart documentclass:
%\usepackage{amsmath}
%\usepackage{amssymb}
%\usepackage{amsthm}

\usepackage{amssymb}

% For dealing with references we use the comment environment
\usepackage{verbatim}
\newenvironment{reference}{\comment}{\endcomment}
%\newenvironment{reference}{}{}
\newenvironment{slogan}{\comment}{\endcomment}
\newenvironment{history}{\comment}{\endcomment}

% For commutative diagrams you can use
% \usepackage{amscd}
\usepackage[all]{xy}

% We use 2cell for 2-commutative diagrams.
\xyoption{2cell}
\UseAllTwocells

% To put source file link in headers.
% Change "template.tex" to "this_filename.tex"
% \usepackage{fancyhdr}
% \pagestyle{fancy}
% \lhead{}
% \chead{}
% \rhead{Source file: \url{template.tex}}
% \lfoot{}
% \cfoot{\thepage}
% \rfoot{}
% \renewcommand{\headrulewidth}{0pt}
% \renewcommand{\footrulewidth}{0pt}
% \renewcommand{\headheight}{12pt}

\usepackage{multicol}

% For cross-file-references
\usepackage{xr-hyper}

% Package for hypertext links:
\usepackage{hyperref}

% For any local file, say "hello.tex" you want to link to please
% use \externaldocument[hello-]{hello}
\externaldocument[introduction-]{introduction}
\externaldocument[representationtheory-]{representationtheory}
\externaldocument[representations-compact-]{representations-compact}
\externaldocument[liegroups-general-]{liegroups-general}
\externaldocument[liestructure-]{liestructure} 
\externaldocument[algebraicgroups-]{algebraicgroups}
\externaldocument[reductiveforms-]{reductiveforms}
\externaldocument[vermamodules-]{vermamodules}
\externaldocument[representations-local-]{representations-local}
%\externaldocument[gKmodules-]{gKmodules}
%\externaldocument[asymptotics-]{asymptotics}
\externaldocument[plancherel-]{plancherel}
\externaldocument[discreteseries-]{discreteseries}
\externaldocument[galoiscohomology-]{galoiscohomology}
\externaldocument[automorphicspace-]{automorphicspace}
%\externaldocument[harmonicanalysis-]{harmonicanalysis} 
%\externaldocument[automorphicforms-]{automorphicforms}
%\externaldocument[periods-]{periods}
%\externaldocument[traceformulalocal-]{traceformulalocal}
%\externaldocument[traceformulaglobal-]{traceformulaglobal}
%\externaldocument[arithmetic-]{arithmetic}
%\externaldocument[geometric-]{geometric}
\externaldocument[fdl-]{fdl}
\externaldocument[index-]{index}

% Theorem environments.
%
\theoremstyle{plain}
\newtheorem{theorem}[subsection]{Theorem}
\newtheorem{proposition}[subsection]{Proposition}
\newtheorem{lemma}[subsection]{Lemma}

\theoremstyle{definition}
\newtheorem{definition}[subsection]{Definition}
\newtheorem{example}[subsection]{Example}
\newtheorem{exercise}[subsection]{Exercise}
\newtheorem{situation}[subsection]{Situation}

\theoremstyle{remark}
\newtheorem{remark}[subsection]{Remark}
\newtheorem{remarks}[subsection]{Remarks}

\numberwithin{equation}{subsection}

% Macros
%
\def\lim{\mathop{\rm lim}\nolimits}
\def\colim{\mathop{\rm colim}\nolimits}
\def\Spec{\mathop{\rm Spec}}
\def\Hom{\mathop{\rm Hom}\nolimits}
\def\SheafHom{\mathop{\mathcal{H}\!{\it om}}\nolimits}
\def\SheafExt{\mathop{\mathcal{E}\!{\it xt}}\nolimits}
\def\Sch{\textit{Sch}}
\def\Mor{\mathop{\rm Mor}\nolimits}
\def\Ob{\mathop{\rm Ob}\nolimits}
\def\Sh{\mathop{\textit{Sh}}\nolimits}
\def\NL{\mathop{N\!L}\nolimits}
\def\proetale{{pro\text{-}\acute{e}tale}}
\def\etale{{\acute{e}tale}}
\def\QCoh{\textit{QCoh}}
\def\Ker{\text{Ker}}
\def\Im{\text{Im}}
\def\Coker{\text{Coker}}
\def\Coim{\text{Coim}}

\def\eqref #1{(\ref{#1})}
\newcommand{\sslash}{\mathbin{/\mkern-6mu/}}


% OK, start here.
%
\begin{document}

\title{Structure theory of Lie algebras}


\maketitle

\phantomsection
\label{section-phantom}

\tableofcontents


\section{Nilpotent and semisimple Lie algebras}

We need at least one reference \cite{reference} in each chapter.

\subsection{Definitions}

\subsection{Engel's theorem}

\subsection{Lie's theorem}

\subsection{Jordan decomposition}

\section{Borel and Cartan subalgebras}


\section{Representations of $\mathfrak{sl}_2(\mathbb C)$.}

\subsection{The Lie algebra $\mathfrak g=\mathfrak{sl}_2(\mathbb C)$, and the center of $U(\mathfrak g)$.}

The Lie algebra of $\mathfrak{sl}_2$ can be identified with the algebra of $2\times 2$ matrices of trace zero, with Lie bracket the commutator of two matrices. It is generated over the underlying field by three elements $H, E, F$ with bracket relations:
$$[H,E]=2E,$$
$$[H,F]=-2F,$$
$$[E,F] = H.$$

It is easily verified that the center $\mathcal Z(\mathfrak g)$ of the universal enveloping algebra contains the element:
$$\Delta = 4FE + (H+2) H.$$

It turns out (but we won't use it -- see the Harish-Chandra isomorphism in later lecture) that $\mathcal Z(\mathfrak g)$ is a polynomial ring generated by this element.

In this lecture, all vector spaces are finite dimensional.

\subsection{Highest weight vectors}

Given a representation $V$ of $\mathfrak{sl}_2$, and $\lambda\in \mathbb C$, let $V_\lambda$ denote the $\lambda$-eigenspace of $H$. We don't know yet that $H$ acts semisimply, so a priori $V$ is not the direct sum of the $V_\lambda$'s. 

\begin{lemma}
 $E\cdot V_\lambda\subset V_{\lambda+2}$; $F\cdot V_\lambda\subset V_{\lambda-2}$.
\end{lemma}

\begin{proposition}
 There is a non-zero vector $v\in V$ which is an eigenvector for $H$ and such that $Ev=0$.
\end{proposition}

This follows by the fact that $V$ is finite dimensional, and the $V_\lambda$'s are linearly independent.

We call $v$ a \emph{highest weight vector}. Similarly, there will be a lowest weight vector, i.e.\ a nonzero eigenvector of $H$ which is annihilated by $F$.

\begin{proposition}\label{sl2prop}
 Fix a heighest weight vector $v\in V_\lambda$, and let $V'$ be the span of $\{F^iv\}_{i\in\mathbb N}$. Then $V'$ is $\mathfrak{sl}_2$-stable, irreducible, and $\Delta$ acts by $\lambda(\lambda_2)$. The highest weight $\lambda$ is a non-negative integer, and $V'$ is the sum of one-dimensional weight spaces $V'_{\mu}$ for $\mu = \lambda, \lambda-2,\lambda-4, \dots, -\lambda$.
\end{proposition}

\begin{proof}
 It is clearly stable under $F$ and $H$. We easily compute:
$$ EF^nv_\lambda = n(\lambda-(n-1))F^{n-1}v_\lambda.$$
Hence, the space is $E$-stable. 

Moreover, since it is finite-dimensional, we must have $n(\lambda-(n-1))=0$ for some $n\ge 1$, hence $\lambda$ is a non-negative integer. In that case, $n=\lambda+1$, and $F^n v_\lambda$ must be zero (because it is a highest weight vector of weight $-\lambda-2$ and, by the same argument, it cannot generate a finite-dimensional representation). On the other hand, for $n<\lambda+1$ $EF^nv_\lambda\ne 0$, hence $F^nv_\lambda\ne 0$. The statement about the weight spaces of $V'$ follows.

We have: $\Delta v = 4FEv+(H+2)Hv = 0+ \lambda(\lambda+2)v$. Since $\Delta$ commutes with the action of $\mathfrak{sl}_2$ and is generated by $v$, all elements of $V'$ have the same $\Delta$-eigenvalue.

On the other hand, $V'$ has at most one eigenvector for each $H$-eigenvalue. If $V'$ was reducible, there would be some highest weight vector with eigenvalue $\ne $
\end{proof}

\begin{proposition}
 Irreducible (finite-dimensional) representations of $\mathfrak{sl}_2$ are $H$-semisimple. 
\end{proposition}

We will eventually see that all finite-dimensional representations of $\mathfrak{sl}_2$ are semisimple, in particular $H$-semisimple.

\begin{lemma}
 For every nonnegative integer $n$ there is an irreducible finite-dimensional representation of heighest weight $n$. It is unique up to isomorphism and has dimension $n+1$.
\end{lemma}

\begin{proof}
 If $V$ denotes the standard, $2$-dimensional representation, then it is easy to see that $S^n V$ has a unique highest weight vector with weight $n+1$, hence is irreducible. Uniqueness follows from the explicit description of the action of $E, F$ and $H$ above.
\end{proof}

This existence statement will require a lot more work in the general case.


\section{Semisimplicity (complete reducibility)}

Suppose that we have a short exact sequence of representations: $0\to A\to B\to C\to 0$. How do we know that it splits? Well, it splits as vector spaces. That is, there is an element of $\Hom_\mathbb C(C,B)$ which lifts the identity element in $\Hom_\mathbb C(C,C)$. We would like to know that there is a $\mathfrak g$-invariant such element. Thus, it suffices to show that if we apply the functor of ``$\mathfrak g$-invariants'' to the exact sequence:
$$0\to \Hom_\mathbb C(C,A)\to \Hom_\mathbb C(C,B) \to \Hom_\mathbb C (C,C) \to 0,$$
it remains exact. 

This is a problem is cohomology. Any short exact sequence of $\mathfrak g$-modules $0\to U\to V\to W\to 0$ (think of the above $\Hom$ spaces here) gives rise to a long exact sequence:
$$0\to U^{\mathfrak g}\to V^{\mathfrak g}\to W^{\mathfrak g}\to H^1(\mathfrak g, U)\to H^1(\mathfrak g, V)\to H^1(\mathfrak g, W)\to \dots.$$

The right derived functor $H^1$ can be explicitly described, it turns out, as follows (more details on Lie algebra cohomology, hopefully, in some future version of these notes):
 $$H^1(\mathfrak g, V) = Z^1(\mathfrak g, V)/C^1(\mathfrak g, V),$$
where the cocycles $Z^1(\mathfrak g, V)$ are maps $f:\mathfrak g\to V$ satisfying:
$$ f([X,Y]) = Xf(Y)-Yf(X),$$
and the coboundaries are those of the form: $f(X)=Xv$ (for some $v\in V$).

\begin{theorem}\label{vanishingcohom}
 For any finite-dimensional $\mathfrak g$-module $V$, $H^1(\mathfrak g,V)=0$.
\end{theorem}

\begin{proof}
 First, we reduce to simple $\mathfrak g$-modules by induction. Suppose that we have a short exact sequence:
$$0\to U\to V\to W\to 0,$$
and that the first cohomology groups of $U$ and $W$ are trivial, then the long exact sequence shows that $H^1(\mathfrak g, V)=0$, as well.

For a simple $V$, if $V=\mathbb C$ then $H^1(\mathfrak g,V)=0$ trivially.

If $V\ne \mathbb C$ then, as we saw in  $\Delta$ acts by a non-zero scalar. Given a cocycle $f$, let $v\in V$ be defined by the equation:
$$\frac{1}{8}\Delta v = \sum_i X_if(Y_i),$$
where $(X_i)_i, (Y_i)_i$ are dual bases for $\mathfrak g$ under the Killing form. We will use the fact that $\Delta = 8(\sum_i X_i Y_i)$, in the case of $\mathfrak{sl}_2$, and in the general case this sum is called the \emph{Casimir} operator, does not depend on the choice of basis and lies in the center of $\mathfrak z(\mathfrak g)$.

One can now easily show that $\Delta f(X)= \Delta Xv = X\Delta v$, which implies that $f(X)=Xv$, for all $X\in \mathfrak g$.
\end{proof}

\begin{proposition}\label{completereducibility}
 Every finite-dimensional $\mathfrak g$-module is semisimple.
\end{proposition}


\section{General $\mathfrak g$}

There is very little that changes in the case of a general $\mathfrak g$. Choosing a Borel subalgebra $\mathfrak b$ and a Cartan subalgebra $\mathfrak h$ thereof, denoting by $(\mathfrak{sl}_2)_\alpha$ the subalgebras $<H_\alpha>+\mathfrak g_\alpha+\mathfrak g_{-\alpha}$, for every positive root $\alpha$, and by $\mathfrak n_-$, $\mathfrak n_+$ the sums of positive/negative weight spaces, we can easily prove as before (namely, using the fact that each nonzero element of $\mathfrak n_+$ raises the weight):

\begin{lemma}
 Every finite dimensional representation $V$ contains a heighest weight vector, i.e.\ an eigenvector for $\mathfrak b$.
\end{lemma}

Obviously, irreducible representations contain a unique highest-weight vector (because the $\mathfrak g$-span of a highest weight vector contains vectors of smaller weight only).

Applying Proposition \ref{sl2prop} to the action of $(\mathfrak{sl}_2)_\alpha$ we get:

\begin{proposition}
 If $\lambda\in \mathfrak h^*$ is the weight of a highest weight vector then $\left<\lambda,\text{ch}eck\alpha\right>\in \mathbb N$ for every root $\alpha>0$. The action of $\mathfrak h$ on a finite-dimensional representation $V$ is semisimple, and for every element $w\in W$ the weight spaces $V_\mu$ and $V_{w\mu}$ have the same dimension.
\end{proposition}

\begin{proof}
The first statement follows directly from Proposition \ref{sl2prop}. 

For the second, let $V'$ the span of $\mathfrak h$-semisimple vectors in the $\mathfrak g$-span of a highest weight vector. I claim that $V'$ is $\mathfrak g$-stable. Indeed, we have a homomorphism of $\mathfrak h$-modules: $\mathfrak g\otimes V'\to V$, and the left hand side is $\mathfrak h$-semisimple. Therefore, its image belongs to $V'$.

Finally, the last statement is enough to prove for simple reflections, and in that case it follows from the analogous statement for $(\mathfrak{sl}_2)_\alpha$-representations.
\end{proof}

We now discuss the Casimir operator $C=\sum_i X_i Y_i$, where $(X_i)_i, (Y_i)_i$ are dual bases with respect to the Killing form. First of all:

\begin{lemma}\label{lemmaCasimir}
 $C$ does not depend on the choice of basis. It belongs to the center of $U(\mathfrak g)$.
\end{lemma}

\begin{proof}
 A better way to define $C$ is as follows: the Killing form is an invariant $2$-tensor on $\mathfrak g^*$, i.e.\ an element of:
$$\left(\otimes^2\mathfrak g^*\right)^{\mathfrak g}.$$
It is nondegenerate and invariant, hence induces an isomorphism of $\mathfrak g$-modules: $\mathfrak g*\to \mathfrak g$. Its image under:
$$\otimes^\bullet \mathfrak g^* \to \otimes^\bullet\mathfrak g \to U(\mathfrak g) $$
is the Casimir operator. This proves the lemma.
\end{proof}

Now:
\begin{lemma}
 The Casimir operator acts on an irreducible representation of highest weight $\lambda$ by the scalar:
$$(\lambda+\rho,\lambda+\rho)-(\rho,\rho),$$
where $(\,\, , \,\,)$ denotes the Killing form and $\rho$ is half the sum of positive roots.
\end{lemma}

\begin{proof}
 The trick is to write the operator as an element of $U(\mathfrak h)$ plus an element in the ideal generated by $\mathfrak n^+$. The latter will kill $v_\lambda$ (the highest weight vector), and the former will give the eigenvalue. 

 We have an orthogonal decomposition: $\mathfrak g = \mathfrak h \oplus \sum_{\alpha>0} (\mathfrak g_\alpha\oplus \mathfrak g_{-\alpha})$; if $X_\alpha \in \mathfrak g_\alpha, Y_\alpha\in \mathfrak g_{-\alpha}$ are dual elements, the Casimir element will be equal to an element $Z\ in U(\mathfrak h)$ plus:
$$\sum_{\alpha>0} (X_\alpha Y_\alpha+Y_\alpha X_\alpha) =  \sum_{\alpha>0}\left( 2 Y_\alpha X_\alpha + [X_\alpha, Y_\alpha]\right).$$
If we apply this to $v_\lambda$ we will get $\lambda\left(\sum_{\alpha>0} [X_\alpha, Y_\alpha]\right)v_\lambda$.

Notice that for $H\in \mathfrak h$, $(H, [X_\alpha,Y_\alpha]) = ([H,X_\alpha], Y_\alpha) = \alpha(H) (X_\alpha,Y_\alpha) = \alpha(H)$, and therefore $[X_\alpha,Y_\alpha]$ is the image of $\alpha$ under the identification: $\mathfrak h^*\to\mathfrak h$ induced by the Killing form. Therefore, 
$$ \lambda\left(\sum_{\alpha>0} [X_\alpha, Y_\alpha]\right) = \sum_{\alpha>0} (\lambda, \alpha) = 2(\lambda,\rho).$$
 

 On the other hand, the element $Z\in U(\mathfrak h)$ is the restriction of the Killing form to $\mathfrak h$, interpreted as in the proof of Lemma \ref{lemmaCasimir}: via the map $\otimes^\bullet \mathfrak h^* \to \otimes^\bullet \mathfrak h\to U(\mathfrak h)$. This means that its evaluation on $\lambda$ is the Killing form $(\lambda,\lambda)$. 

 Hence we get that the Casimir acts on $v_\lambda$ by the scalar: $ (\lambda,\lambda)+2 (\lambda,\rho) = (\lambda+\rho,\lambda+\rho)-(\rho,\rho).$
\end{proof}

Finally, vanishing of cohomology and complete reducibility are proven as previously, i.e.\ Theorem \ref{vanishingcohom} and Proposition \ref{completereducibility} hold for all semisimple $\mathfrak g$.


\begin{remark}
 All the above could be done with an arbitrary nondegenerate invariant bilinear form, not necessarily the Killing form, replacing the Casimir element accordingly. But on each simple factor, such a form is necessarily a multiple of the Killing form. Moreover, Theorem \ref{vanishingcohom} can be proven using the invariant form $(X,Y)\mapsto \text{tr}_V(X,Y)$ which vanishes on the kernel of the representation $V$.
\end{remark}



%************************************************************************

\section{Structure of general (finite dimensional) Lie algebras}


%***************************************************************************

\section{Structure of semisimple Lie algebras}

In fact, the present lecture also contains some general theorems for the existence and conjugacy of Cartan subalgebras and Borel subalgebras, whose proof, however, is eventually reduced to the semisimple case.

\section{Jordan decomposition in $\mathfrak{gl}$.}

\begin{definition}
 Let $\mathfrak g$ be a Lie algebra. An element $X\in \mathfrak g$ is called \emph{semisimple} if $\text{ad}(X)$ is a semisimple operator, and \emph{nilpotent} if $\text{ad}(X)$ is nilpotent.
\end{definition}

\begin{remark}
 For the Lie algebra $\mathfrak g = \text{End}(V)$, where $V$ is a vector space, these notions of semisimple and adjoint do not completely coincide with ``semisimple operator'' and ``nilpotent operator'', but they coincide modulo the center of $\mathfrak g$, as the following result shows. For this case, we will be using the word ``semisimple'' or ``adjoint'' to refer to the property of the operator, unless otherwise stated.
\end{remark}

\begin{proposition}
 An operator $T\in \text{End}(V)$ is semisimple iff the operator $\text{ad}(T)\in \text{End}(\text{End}(V))$ is semisimple. It is nilpotent in $\text{End}(V)/k$ (where $k$ stands for the center of $\text{End}(V)$ iff $\text{ad}(T)$ is nilpotent.
\end{proposition}

Now recall that for any element $X\in \text{End}(V)$ (some vector space $V$) there is a unique \emph{Jordan decomposition} $X=X_s+X_n$ with $X_s$ semisimple, $X_n$ nilpotent and $[X_s,X_n]=0$. (Moreover, $X_s$ and $X_n$ commute with every element commuting with $X$, since they can be expressed as polynomials in $X$.) Moreover, there are polynomials $P,Q\in k^{p^{-\infty}}[T]$ (where $k$ is the base field, $p$ the characteristic) such that $X_s=P(X)$ and $X_n=Q(X)$ -- in particular, \emph{$X_s$ and $X_n$ are defined over $k^{p^{-\infty}}$}. (Indeed, if the characteristic polynomial over the algebraic closure is written $\prod_i (T-a_i)^m_i$, with all $a_i$ distinct, choose $P(T)$ to satisfy the congruences: 
$$ P(T)\equiv a_i \mod (T-a_i)^m_i, \,\, P(T)\equiv 0 \mod T.)$$


\section{Derivations and the Jordan decomposition}

\begin{definition}
 A \emph{derivation} of a Lie algebra $\mathfrak g$ is a linear map $D:\mathfrak g\to\mathfrak g$ satisfying $D([X,Y])=[X,D(Y)]+[D(X),Y]$.
\end{definition}

\begin{remarks}
 \begin{enumerate}
  \item This is a very natural extension of the definition of derivation for an associative algebra, since such a derivation induces a derivation as above on the associated Lie algebra. Vice versa, a derivation of a Lie algebra induces a derivation of its universal enveloping algebra.
  \item Derivations form a Lie subalgebra of $\text{End}(\mathfrak g)$.
  \item The adjoint representation $\text{ad}:\mathfrak g\to \text{End}(\mathfrak g)$ has image in $\text{Der}(\mathfrak g)$. 
 \end{enumerate}
\end{remarks}

\begin{proposition}\label{inner}
 Every derivation of a \emph{semisimple} Lie algebra is \emph{inner}, i.e.\ in the image of $\text{ad}$.
\end{proposition}

\begin{proof}
 The formula $[D,\text{ad}(X)]=\text{ad}(DX)$ shows that the image of $\text{ad}$ is an ideal in $\text{Der}(\mathfrak g)$. Since the image is a semisimple Lie algebra, there is a complementary ideal $I$ (namely, its orthogonal complement under the Killing form on $\text{Der}(\mathfrak g)$). But if $D\in I$, and $I$ is an ideal, the same formula shows that $\text{ad}(DX)\in I\cap \text{ad}(\mathfrak g)=0$, which since $\text{ad}$ is injective means that $DX=0$, i.e.\ $D=0$.
\end{proof}


\begin{proposition}
 The identity component of the automorphism group of a semisimple Lie group coincides with the group of inner automorphisms.
\end{proposition}



\begin{proposition}\label{isderivation}
 If $D\in \text{Der}(\mathfrak g)$ then $D_s,D_n\in\text{Der}(\mathfrak g)$.
\end{proposition}

\begin{proof}
 If $X$ is in the generalized $\lambda$-eigenspace and $Y$ is in the generalized $\mu$-eigenspace for $D$, then it can be shown by induction that:
$$ (D-(\lambda+\mu))^n([X,Y]) = \sum_{r=0}^n \binom{n}{r} [(D-\lambda)^r(X),(D-\mu)^{n-r}(Y)],$$
hence $[X,Y]$ is in the generalized $\mu+\lambda$-eigenspace. This shows that $D_s$ is a derivation, and then $D_n=D-D_s$ is a derivation.
\end{proof}

\begin{theorem}
 Let $\mathfrak g$ be a semisimple Lie algebra. Then every element has a unique decomposition (over $k^{p^{-\infty}}$), $X=X_s+X_n$ with $X_s$ semisimple, $X_n$ nilpotent and $[X_s,X_n]=0$. 
\end{theorem}

\begin{proof}
 By the previous two propositions, $\text{ad}(X)_s$ and $\text{ad}(X)_n$ are derivations and therefore belong to the image of $\text{ad}$. This proves the existence (and uniqueness) of $X_s$ and $X_n$. 
\end{proof}

If we assume complete reducibility of finite-dimensional representations of semisimple Lie algebras (which we will prove later), we can show that the Jordan decomposition is preserved by homomorphisms of semisimple Lie algebras. In fact, we can show more generally:

\begin{theorem}
 For a homomorphism $\rho:\mathfrak g\to \text{End}(V)$ we have $\rho(X_s)=\rho(X)_s$, $\rho(X_n)=\rho(X)_n$.
\end{theorem}

\begin{proof}
 By complete reducibility of $\text{End}(V)$ under the adjoint $\mathfrak g$-action, we have:
$$\text{End}(V)= \rho(\mathfrak g)\oplus\mathfrak m,$$
where $\mathfrak m$ is an $\text{ad}(\rho(\mathfrak g))$-invariant subspace. (Notice that in Proposition \ref{inner} we were able to obtain a similar decomposition in $\text{Der}(\mathfrak g)$ by using the Killing form, so we did not need to know reducibility.)

Since $\rho(X)_s,\rho(X)_n$ are polynomials in $\rho(X)$, their adjoint action preserves both $\rho(\mathfrak g)$ and $\mathfrak m$. Let $\rho(X)_n = \rho(a) + b$ with $a\in \mathfrak g,b\in \mathfrak b$. Then $[\rho(\mathfrak g),b]=0$, which means that $b\in \text{End}(V)$ is a $\mathfrak g$-endomorphism. If $V=\oplus V_i$ is a decomposition into irreducibles, $b$ acts by a scalar on each one of them, by Schur's lemma. On the other hand, we know that $\rho(X)_n$ is nilpotent, $\rho(a)$ and $b$ commute, and $\text{tr}_{V_i}(\rho(a)) =0$ because $a$ (like every element of $\mathfrak g$) is a sum of commutators. Therefore, $\text{tr}_{V_i}(b)=0$, hence $b$ acts by zero on all $V_i$, i.e.\ $b=0$. 

Now, $\rho(X)_n=\rho(a)$ acts nilpotently on $V$, hence it acts nilpotently on $\text{End}(V)$ under the adjoint representation. By the decomposition $\text{End}(V)=\mathfrak g\oplus \mathfrak m$ it follows that it acts nilpotently on $\mathfrak g$. By the uniqueness of the Jordan decomposition we can now infer that $\rho(X)_n= \rho(X_n)$. 
\end{proof}



We will see later that the image of a semisimple element of $\mathfrak g$ (where $\mathfrak g$ is semisimple) under any representation is a semisimple operator, and the image of a nilpotent element is a nilpotent operator.



\section{Cartan subalgebras}

A \emph{Cartan subalgebra} of a Lie algebra $\mathfrak g$ is a \emph{nilpotent, self-normalizing} subalgebra $\mathfrak h$. Here $\mathfrak g$ is not (yet) necessarily semisimple.

We will construct Cartan subalgebras as nilspaces (generalized eigenspaces of zero under the adjoint representation) of \emph{s-regular} elements. Then we will show that they are all conjugate to each other.

\begin{definition}
 An s-regular element $X\in\mathfrak g$ is an element with minimal possible $0$-generalized eigenspace under $\text{ad}$.
\end{definition}

Since the dimension of the zero generalized eigenspace is the heighest power of $t$ which divides the characteristic polynomial of $\text{ad}(X)$, it follows that s-regular elements form a (nonempty) Zariski open set.

\begin{remark}
In many books on Lie algebras, the word ``regular'' is used for ``s-regular'' (which is my invention!). The problem with this is that it has become nowadays standard to call ``regular'' the elements with minimal \emph{zero eigenspace}, i.e.\ \emph{centralizer}, instead of generalized eigenspace. This includes non-semisimple elements, while s-regular, as we shall see, implies semisimple (for semisimple Lie algebras) -- hence for those: s-regular = regular semisimple.
\end{remark}

\begin{proposition}
 The nilspace of a s-regular element is a Cartan subalgebra.
\end{proposition}

\begin{proof}
 Let $X$ be the s-regular element and $\mathfrak h$ its centralizer. We will prove that $\mathfrak h$ is nilpotent; equivalently, by Engel's theorem, that the restriction of $\text{ad}(Y)$ to $\mathfrak h$, for any $Y\in\mathfrak h$, is nilpotent. Let $U\subset\mathfrak h$ be the subset of elements which fail to satisfy this; it is a Zariski open subset (again by considerations of the characteristic polynomial). Let $V\subset\mathfrak h$ be the subset of elements which act invertibly on $\mathfrak g/\mathfrak h$. It is again a Zariski open subset, and non-empty since $X\in V$. If $U\ne\emptyset$ then $U\cap V\ne\emptyset$, i.e.\ there exists an element $Y\in\mathcal h$ such that the dimension of the zero generalized eigenspace for $Y$ is less than the dimension of $\mathfrak h$, a contradiction by the s-regularity of $X$. Thus, $\mathfrak h$ is nilpotent.

 If $Z$ normalizes $\mathfrak h$ then $[Z,X]\in\mathfrak h$ which implies that $Z$ is in the nilspace of $X$, i.e.\ in $\mathfrak h$.
\end{proof}

Hence, every Lie algebra has Cartan subalgebras. We will eventually prove that any two Cartan subalgebras are conjugate (over the algebraic closure) by the group of inner automorphisms of $\mathfrak g$ and, in particular, equal to nilspaces of s-regular elements.


\section{Root decomposition, semisimple case}

\begin{theorem}
 Assume that $\mathfrak g$ is semisimple, and let $\mathfrak h$ be the nilspace of an s-regular element (hence\footnote{Eventually, since they are conjugate, all Cartan subalgebras are of this form} a Cartan subalgebra). Then:
\begin{enumerate}
 \item $\mathfrak h$ is abelian.
 \item The centralizer of $\mathfrak h$ is $\mathfrak h$.
 \item Every element of $\mathfrak h$ is semisimple.
 \item The restriction of the Killing form (or any non-degenerate invariant form) of $\mathfrak g$ to $\mathfrak h$ is non-degenerate.
\end{enumerate}
\end{theorem}

\begin{proof}
 The rest of the statements follow from the last one. Let us see how: 

Cartan's criterion says that a Lie subalgebra $\mathfrak a$ of $\text{End}(V)$ is solvable if and only if $\text{tr}(XY)=0$ for every $X\in \mathfrak a, Y\in [\mathfrak a,\mathfrak a]$. Applying this to $\text{ad}(\mathfrak h)\subset \text{End}(\mathfrak g)$ (which is nilpotent, hence solvable), we get that $B(X,Y)=0$ for all $X\in \mathfrak h, Y\in[\mathfrak h,\mathfrak h]$ (where $B$ is the Killing form for $\mathfrak g$). Therefore, the radical of the restriction of $B$ to $\mathfrak h$ contains the commutator, which means that $[\mathfrak h,\mathfrak h]=0$.

The centralizer is contained in the normalizer, which is $\mathfrak h$, but since $\mathfrak h$ is abelian it coincides with it.

Finally, let $X\in\mathfrak h$ and let $X=X_s+X_n$ be its Jordan decomposition. Since $X_s, X_n$ commute with the centralizer of $X$, which contains $\mathfrak h$, it follows that $X_s, X_n$ are in the centralizer of $\mathfrak h$, which is $\mathfrak h$. Thus, if $Y\in\mathfrak h$, $\text{ad}(Y)\text{ad}(X_n)$ is nilpotent, which implies that $\text{ad}(X_n)$ is orthogonal to $\mathfrak h$ under the Killing form. By non-degeneracy of the Killing form on $\mathfrak h$, $X_n=0$.

We come to the proof of the last statement: if $X$ is a regular element such that $\mathfrak g$ is (the nilspace of $X$), let $\mathfrak g = \bigoplus_\lambda \mathfrak g_\lambda$ be a decomposition of $\mathfrak g$ into generalized $\text{ad}(X)$-eigenspaces. As we saw in the proof of Proposition \ref{isderivation}, $[\mathfrak g_\lambda,\mathfrak g_\mu]\subset \mathfrak g_{\lambda+\mu}$, which implies that $\mathfrak g_\lambda \perp \mathfrak g_\mu$ (under the Killing form), unless $\lambda+\mu=0$. Therefore, the decomposition:
$$ \mathfrak g = \mathfrak g_0 \oplus \bigoplus (\mathfrak g_{\lambda}\oplus \mathfrak g_{-\lambda})$$
is orthogonal, and since $B$ is nondegenerate, it has to be non-degenerate on each of the summands, in particular on $\mathfrak h=\mathfrak g_0$.
\end{proof}


\begin{proposition}
 $\mathfrak h$ is a maximal abelian subalgebra of $\mathfrak g$. Every s-regular element is semisimple.
\end{proposition}


The decomposition of $\mathfrak g$ into generalized eigenspaces for the adjoint action of a Cartan subalgebra $\mathfrak h$ is called the \emph{Cartan decomposition} of $\mathfrak g$:
$$\mathfrak g = \mathfrak h + \sum_{\alpha\Phi\subset\mathfrak h^*} \mathfrak g_\alpha.$$
The set $\Phi$ of non-zero elements of $\mathfrak h^*$ which appear in this decomposition is called the set of \emph{roots} of $\mathfrak g$.



Now I collect quickly the basic facts about the root decomposition, which are well-known and easy to follow in the literature. Only hints about the proofs are given.

\begin{itemize}
 \item $[\mathfrak g_\alpha,\mathfrak g_\beta]\subset\mathfrak g_{\alpha+\beta}$.
 \item $\Phi$ spans $\mathfrak h^*$.
 \item If $\alpha\in\Phi$, let $\mathcal h_\alpha \in \mathfrak h$ be the image of $\alpha$ under the isomorphism: $\mathfrak h^*\to\mathfrak h$ defined by a non-degenerate invariant form $(\,,\,)$ on $\mathfrak g$. Then, for all $H\in \mathfrak h$, $X\in \mathfrak g_\alpha, Y\in\mathfrak g_{-\alpha}$ we have: $(H,[X,Y]) = ([H,X],Y) = \alpha(H) (X,Y) = (H, (X,Y) h_\alpha)$, and therefore: $[X,Y] = (X,Y) h_\alpha$. Notice that the pairing $(\,,\,)$ between $\mathfrak g_\alpha$ and $\mathfrak g_{-\alpha}$ is nonzero because otherwise it would be degenerate on $\mathfrak g$. 

It follows that $\mathfrak h_\alpha:= [\mathfrak g_\alpha,\mathfrak g_{-\alpha}]$ is one-dimensional, spanned by the element $h_\alpha$. We let $H_\alpha$ (or $\text{ch}eck\alpha$) denote its unique multiple with $\alpha(H_\alpha)=2$ (the \emph{coroot} associated to $\alpha$).

 \item The sum $\mathfrak h_\alpha+ \mathfrak g_\alpha + \mathfrak g_{-\alpha}$ is a subalgebra isomorphic to $\mathfrak{sl}_2$. To prove this, one first shows that it includes an $\mathfrak{sl}_2$-triple $(H,E,F)$. An element of $\mathfrak g_{-\alpha}$ orthogonal to $E$ would be a highest weight vector of weight $-2$, which is absurd. Therefore, $\mathfrak g_{-\alpha}$ is one-dimensional, and so is $\mathfrak g_\alpha$.

 \item If $\alpha, c\alpha\in \Phi$ then $c = \pm 1$.

 \item For $\alpha\in \Phi$, the reflection $\mathfrak h^*\ni \gamma \mapsto \gamma - \left<\gamma,H_\alpha\right> \alpha$ fixes $\Phi$.

Indeed, this follows from viewing $\mathfrak g$ as an $\mathfrak{sl}_2$-module: a nonzero element $Y$ of $\mathfrak g_\gamma$ has weight $\left<\gamma,H_\alpha\right>$, and by properties of $\mathfrak{sl}_2$-modules the element $F^pY$ is nonzero. But that lives in $\mathfrak g_{\gamma - \left<\gamma,H_\alpha\right> \alpha}$.
\end{itemize}




\section{Conjugacy of Borel subalgebras and the universal Cartan: statements}


A maximal solvable subalgebra of a Lie algebra (over the algebraic closure) is called a \emph{Borel subalgebra}. Clearly, a Borel subalgebra contains the radical of $\mathfrak g$, and therefore most questions (most importantly, the question of conjugacy) are reduced to the semisimple case.

The following is immediate:
\begin{lemma}
 Every Borel subalgebra is its own normalizer.
\end{lemma}


Let $G$ be the group of \emph{inner automorphisms} of $\mathfrak g$, i.e.\ automorphisms of the form $\exp(\text{ad}(y))\in {\rm GL}(\mathfrak g)$. Here $\mathfrak g$ is not assumed to be semisimple, although clearly the proof of the following theorem reduces to the semisimple case:

\begin{theorem}
 All Borel subalgebras are $G$-conjugate.
\end{theorem}

\begin{remark}
 The definition of $G$ is not satisfactory, because it relies on the exponential map which is not algebraic. Here is how we would define an algebraic subgroup of ${\rm GL}(V)$ in general, in characteristic zero: We would take $G$ to be the subgroup generated by $\exp(\text{ad}(X))$ for all \emph{nilpotent} elements of $X$ (the exponential is algebraic on them!). It turns out, for semisimple groups at least (which is all we care about in this theorem) that these generate the same group as the one analytically defined above. In arbitrary characteristic, over an algebraically closed field $k$, an algebraic group $G(k)$ acts transitively on the Borel subalgebras of its Lie algebra. 
\end{remark}

The proof will use many lemmas, including:

\begin{lemma}
 All Cartan subalgebras of a solvable Lie algebra are conjugate.
\end{lemma}

(We write ``conjugate'' for the group of inner automorphisms of the given Lie algebra, i.e.\ in the last lemma the inner automorphisms of the solvable algebra. Again, exponentials of nilpotent elements will suffice, although in this case they do not generate the whole group of inner automorphisms.)

\begin{proposition}
 All Cartan subalgebras of a Lie algebra $\mathfrak g$ are conjugate  (over the algebraic closure).
\end{proposition}

\begin{proof}[Proof of the proposition]
 Any Cartan subalgebra is nilpotent, hence solvable, hence contained (over the algebraic closure) in a Borel subalgebra. Two Borel subalgebras are conjugate, and two Cartan subalgebras of a given Borel are conjugate.
\end{proof}

\begin{definition}
 The \emph{universal Cartan} $\mathfrak h$ of $\mathfrak g$ is the quotient of any Borel subalgebra $\mathfrak b$ by its nilpotent radical. It is a commutative Lie algebra. For two different Borel subalgebras $\mathfrak b$ and $\mathfrak b'$, we identify the corresponding quotients $\mathfrak h$ and $\mathfrak h'$ by picking an element $g\in G$ which conjugates $\mathfrak b$ to $\mathfrak b'$; since $\mathfrak b$ is its own normalizer, such an identification is unique up to an inner automorphism of $\mathfrak b$, but inner automorphisms act trivially on the quotient $\mathfrak h$; therefore, \emph{$\mathfrak h$ is unique up to unique isomorphism}.
\end{definition}

The universal Cartan comes with a set of roots $\Phi\subset \mathfrak h^*$, and in fact \emph{with a canonical choice of positive roots} $\Phi^+$ (those that appear in the decomposition of $\mathfrak b$). This will play a role in the definition of the Langlands dual group, which has, by definition a canonical maximal torus with Lie algebra isomorphic to $\mathfrak h^*$.

\section{The scheme of Borel subgroups}

A maximal solvable subgroup of an algebraic group over an algebraically closed field $k$ is called a \emph{Borel subgroup}. For a smooth group scheme $G$ (of finite presentation) over an arbitrary base $S$, we call \emph{Borel subgroup} any smooth subgroup scheme of finite presentation $B$ of $G$ over $S$ such that for all $s\in S$ the geometric fiber $B_{\bar s}$ is a Borel subgroup (SGA3, XXIV, 4.5). 

In general, it is not true that the functor which assigns to any $S$-scheme $T$ the set of Borel subgroups over $T$ is reprentable, i.e.\ that there exists a scheme $\mathcal B$ over $S$ such that $\mathcal B(T)$ is the set of Borel subgroups over $T$. For instance, $G$ would act by automorphisms on such a scheme, and the kernel would be equal to the radical of $G$, but if $S=\text{spec} k$ and $k$ is not perfect then the radical may not be defined over $k$.

However, if $G$ is reductive then it is proven in SGA3, XXVI, 3.3:
\begin{theorem}
If $G$ is a reductive subgroup (i.e.\ smooth, and all geometric fibers are reductive) over a base $S$, the functor which assigns to an $S$-scheme $T$ the set of Borel subgroups of $G$ over $T$ is representable by a smooth, projective, geometrically integral group scheme $\mathfrak B$ over $S$.  
\end{theorem}

If there is a Borel subgroup $B$ over $S$ then $\mathcal B\simeq G/B$ under the action of $G$. The implications of this theorem are very important; for instance:

\begin{proposition} If $G$ is a reductive group defined over a global field $k$, then it is quasi-split (i.e.\ has a Borel) over almost all completions $k_v$ of $k$. 
\end{proposition}

\begin{proof} 
Indeed, for a finite set of places $V$ there is a reductive model of $G$ over the $V$-integers $\mathfrak o_V$. Consider the scheme $\mathfrak B$ of Borel subgroups over $S=\text{spec} \mathfrak o_V$. Let $R$ be a non-archimedean completion of $\mathfrak o_V$ and let $F$ be the residue field of $R$ (a finite field). Since $G$ is reductive over $F$, it is known that it has a Borel subgroup defined over $F$. In other words, the scheme $\mathcal B$ has an $F$-point. A smooth scheme is \emph{formally smooth}, which in our case implies the henselian property: every $F$-point of $\mathcal B$ lifts to an $R$-point. In other words, there is a Borel subgroup at every non-archimedean place outside of $V$.
\end{proof}


\section{Positive roots and standard Borel subgroups}

We return to working over an algebraically closed field. 

The following are combinatorial properties of root systems:

\begin{proposition}
 Let $(V,\Phi,s_\alpha)$ be a root system, let $l$ be a functional which does not vanish on $\Phi$ and let $\Phi^+$ be the elements of $\Phi$ on which $l$ is positive. This is called a choice of \emph{positive roots}. Let $\Delta$ denote the set of elements of $\Phi^+$ which cannot be written as sums of other elements of $\Phi^+$ with positive integral coefficients. (These are the \emph{simple roots} for this choice of positive roots.) Then: 
\begin{enumerate}
 \item Every $\alpha\in\Phi^+$ is a sum of elements of $\Delta$.
 \item If $\alpha,\beta\in\Delta$ then $(\alpha,\beta)\le 0$.
 \item The elements of $\Delta$ are linearly independent.
\end{enumerate}

\begin{proof}
 The first property follows immediately from the definition of $\Delta$. The second is proved by classifying root systems of rank 2. For the third, if we had a linear relation $\sum_{\alpha_i\in \Delta} c_i\alpha_i=0$ then some of the $c_i$'s have to be negative, say for $i=1,\dots,r$. Then: $$\sum_{i=1}^r (-c_i)\alpha_i= \sum_{i=r+1}^{|\Delta|} c_i\alpha_i$$
$$\Rightarrow \left\|\sum_{i=1}^r (-c_i)\alpha_i\right\|^2 = \left<\sum_{i=1}^r (-c_i)\alpha_i, \sum_{i=r+1}^{|\Delta|} c_i\alpha_i\right>$$
which by the second statement is less or equal to zero, a contradiction.
\end{proof}


\end{proposition}


We call \emph{standard Borel subalgebra} a subalgebra of the form $\mathfrak h + \sum_{\alpha\in\Phi^+} \mathfrak g_\alpha$, where $\mathfrak h$ is a Cartan subalgebra and $\Phi^+$ is some choice of positive roots. Of course, in the end it will turn out that every Borel subalgebra is standard.

The proof of conjugacy of Borel subalgebras goes by decreasing induction on the dimension of $\mathfrak b\cap \mathfrak b'$, where $\mathfrak b$ is assumed to be standard, the case where one is included in the other being obvious. 

[TO BE CONTINUED]


\section{The Harish-Chandra isomorphism}


\section{Jacobson--Morosov}

%***************************************************************************




\begin{multicols}{2}[\section{Other chapters}]
\noindent
\begin{enumerate}
\item \hyperref[introduction-section-phantom]{Introduction}
\item \hyperref[representationtheory-section-phantom]{Basic Representation Theory}
\item \hyperref[representations-compact-section-phantom]{Representations of compact groups}
\item \hyperref[liegroups-general-section-phantom]{Lie groups and Lie algebras: general properties}
\item \hyperref[liestructure-section-phantom]{Structure of finite-dimensional Lie algebras}
\item \hyperref[algebraicgroups-section-phantom]{Linear algebraic groups}
\item \hyperref[reductiveforms-section-phantom]{Forms and covers of reductive groups, and the $L$-group}
\item \hyperref[vermamodules-section-phantom]{Verma modules}
\item \hyperref[representations-local-section-phantom]{Representations of reductive groups over local fields}
%\item \hyperref[gKmodules-section-phantom]{$(\mathfrak g, K)$-modules}
%\item \hyperref[asymptotics-section-phantom]{Asymptotics and the Langlands classification}
\item \hyperref[plancherel-section-phantom]{Plancherel formula: reduction to discrete spectra}
\item \hyperref[discreteseries-section-phantom]{Construction of discrete series}
\item \hyperref[galoiscohomology-section-phantom]{Galois cohomology of linear algebraic groups}
\item \hyperref[automorphicspace-section-phantom]{The automorphic space}
%\item \hyperref[harmonicanalysis-section-phantom]{Harmonic analysis over local fields}
%\item \hyperref[automorphicforms-section-phantom]{Automorphic forms}
%\item \hyperref[periods-section-phantom]{Periods, theta correspondence, related methods}
%\item \hyperref[traceformulalocal-section-phantom]{The trace formula: local aspects}
%\item \hyperref[traceformulaglobal-section-phantom]{The trace formula: global aspects}
%\item \hyperref[arithmetic-section-phantom]{Arithmetic, reciprocity, Shimura varieties}
%\item \hyperref[geometric-section-phantom]{Geometric aspects}
\item \hyperref[fdl-section-phantom]{GNU Free Documentation License}
\item \hyperref[index-section-phantom]{Auto Generated Index}
\end{enumerate}
\end{multicols}





\bibliography{my}
\bibliographystyle{amsalpha}

\end{document}
