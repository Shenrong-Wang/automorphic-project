\IfFileExists{stacks-project.cls}{%
\documentclass{stacks-project}
}{%
\documentclass{amsart}
}

% The following AMS packages are automatically loaded with
% the amsart documentclass:
%\usepackage{amsmath}
%\usepackage{amssymb}
%\usepackage{amsthm}

\usepackage{amssymb}

% For dealing with references we use the comment environment
\usepackage{verbatim}
\newenvironment{reference}{\comment}{\endcomment}
%\newenvironment{reference}{}{}
\newenvironment{slogan}{\comment}{\endcomment}
\newenvironment{history}{\comment}{\endcomment}

% For commutative diagrams you can use
% \usepackage{amscd}
\usepackage[all]{xy}

% We use 2cell for 2-commutative diagrams.
\xyoption{2cell}
\UseAllTwocells

% To put source file link in headers.
% Change "template.tex" to "this_filename.tex"
% \usepackage{fancyhdr}
% \pagestyle{fancy}
% \lhead{}
% \chead{}
% \rhead{Source file: \url{template.tex}}
% \lfoot{}
% \cfoot{\thepage}
% \rfoot{}
% \renewcommand{\headrulewidth}{0pt}
% \renewcommand{\footrulewidth}{0pt}
% \renewcommand{\headheight}{12pt}

\usepackage{multicol}

% For cross-file-references
\usepackage{xr-hyper}

% Package for hypertext links:
\usepackage{hyperref}

% For any local file, say "hello.tex" you want to link to please
% use \externaldocument[hello-]{hello}
\externaldocument[introduction-]{introduction}
\externaldocument[representationtheory-]{representationtheory}
\externaldocument[representations-compact-]{representations-compact}
\externaldocument[liegroups-general-]{liegroups-general}
\externaldocument[liestructure-]{liestructure} 
\externaldocument[algebraicgroups-]{algebraicgroups}
\externaldocument[reductiveforms-]{reductiveforms}
\externaldocument[vermamodules-]{vermamodules}
\externaldocument[representations-local-]{representations-local}
%\externaldocument[gKmodules-]{gKmodules}
%\externaldocument[asymptotics-]{asymptotics}
\externaldocument[plancherel-]{plancherel}
\externaldocument[discreteseries-]{discreteseries}
\externaldocument[galoiscohomology-]{galoiscohomology}
\externaldocument[automorphicspace-]{automorphicspace}
%\externaldocument[harmonicanalysis-]{harmonicanalysis} 
%\externaldocument[automorphicforms-]{automorphicforms}
%\externaldocument[periods-]{periods}
%\externaldocument[traceformulalocal-]{traceformulalocal}
%\externaldocument[traceformulaglobal-]{traceformulaglobal}
%\externaldocument[arithmetic-]{arithmetic}
%\externaldocument[geometric-]{geometric}
\externaldocument[fdl-]{fdl}
\externaldocument[index-]{index}

% Theorem environments.
%
\theoremstyle{plain}
\newtheorem{theorem}[subsection]{Theorem}
\newtheorem{proposition}[subsection]{Proposition}
\newtheorem{lemma}[subsection]{Lemma}

\theoremstyle{definition}
\newtheorem{definition}[subsection]{Definition}
\newtheorem{example}[subsection]{Example}
\newtheorem{exercise}[subsection]{Exercise}
\newtheorem{situation}[subsection]{Situation}

\theoremstyle{remark}
\newtheorem{remark}[subsection]{Remark}
\newtheorem{remarks}[subsection]{Remarks}

\numberwithin{equation}{subsection}

% Macros
%
\def\lim{\mathop{\rm lim}\nolimits}
\def\colim{\mathop{\rm colim}\nolimits}
\def\Spec{\mathop{\rm Spec}}
\def\Hom{\mathop{\rm Hom}\nolimits}
\def\SheafHom{\mathop{\mathcal{H}\!{\it om}}\nolimits}
\def\SheafExt{\mathop{\mathcal{E}\!{\it xt}}\nolimits}
\def\Sch{\textit{Sch}}
\def\Mor{\mathop{\rm Mor}\nolimits}
\def\Ob{\mathop{\rm Ob}\nolimits}
\def\Sh{\mathop{\textit{Sh}}\nolimits}
\def\NL{\mathop{N\!L}\nolimits}
\def\proetale{{pro\text{-}\acute{e}tale}}
\def\etale{{\acute{e}tale}}
\def\QCoh{\textit{QCoh}}
\def\Ker{\text{Ker}}
\def\Im{\text{Im}}
\def\Coker{\text{Coker}}
\def\Coim{\text{Coim}}

\def\eqref #1{(\ref{#1})}
\newcommand{\sslash}{\mathbin{/\mkern-6mu/}}


 \DeclareFontFamily{U}{wncy}{}
    \DeclareFontShape{U}{wncy}{m}{n}{<->wncyr10}{}
    \DeclareSymbolFont{mcy}{U}{wncy}{m}{n}
    \DeclareMathSymbol{\Sha}{\mathord}{mcy}{"58} 

% OK, start here.
%
\begin{document}

\title{Galois cohomology of linear algebraic groups}


\maketitle

\phantomsection
\label{section-phantom}

\tableofcontents

[This chapter needs a lot of work. For now, we only summarize the results needed in other sections.]

A good reference for this section is Chapter 6 of \cite{Platonov-Rapinchuk}

\section{Galois cohomology over a finite field}
\label{section-finite-field}

\begin{theorem}[Lang's theorem]
 \label{theorem-Lang}
If $G$ is a connected algebraic group over a finite field $k$, then $H^1(k, G)=1$. 
\end{theorem}

\begin{proof}
 
\end{proof}


\section{Tate--Nakayama duality for tori}
 \label{section-Tate-Nakayama}
 

\section{Cohomology of reductive groups over local fields}
\label{section-local-fields}

\begin{lemma}
 \label{lemma-cohomology-finite}
If $G$ is an algebraic group over a local field $F$, then $H^1(F, G)$ is finite.
\end{lemma}

\begin{proof}
 
\end{proof}


\begin{theorem}
 \label{theorem-H1-trivial}
If $G$ is a (connected) simply connected, semisimple group over a non-Archimedean field $F$, then $H^1(F,G)$ is trivial. For an arbitrary connected reductive group over a local field $F$, there is a canonical surjective map [Kottwitz], which in the non-Archimedean case is a bijection.
\end{theorem}

\begin{proof}
 
\end{proof}

\section{Cohomology of reductive groups over global fields; the Hasse principle}
\label{section-global-fields}

\begin{definition}
 \label{definition-Sha-Hasse}
Let $G$ be an algebraic group over a global field $k$. The kernel of the natural map $H^1(k,G)\to \prod_v H^1(k_v, G)$ is the {\it Tate--Shafarevich group} of $G$, denoted $\Sha(G)$. We say that $G$ satisfies the {\it Hasse principle} if $\Sha(G)=1$.
\end{definition}



\begin{theorem}
 \label{theorem-Sha-finite}
 If $G$ is an algebraic group over a number field, then $\Sha(G)$ is finite.
\end{theorem}

\begin{proof}
 
\end{proof}


The following is the Hasse principle for algebraic groups:

\begin{theorem}
 \label{theorem-Hasse-principle}
If $G$ is (connected and) simply connected or adjoint, over a global field, then $\Sha(G) =1$.
\end{theorem}

\begin{proposition}
 \label{proposition-H1-surjects}
If $G$ is a connected algebraic group over a number field $k$, then $H^1(k,G) \to \prod_{v|\infty} H^1(k_v, G)$ is surjective.
\end{proposition}


\begin{proof}
 See \cite[Proposition 6.17]{Platonov-Rapinchuk}.
\end{proof}






%***************************************************************************




\begin{multicols}{2}[\section{Other chapters}]
\noindent
\begin{enumerate}
\item \hyperref[introduction-section-phantom]{Introduction}
\item \hyperref[representationtheory-section-phantom]{Basic Representation Theory}
\item \hyperref[representations-compact-section-phantom]{Representations of compact groups}
\item \hyperref[liegroups-general-section-phantom]{Lie groups and Lie algebras: general properties}
\item \hyperref[liestructure-section-phantom]{Structure of finite-dimensional Lie algebras}
\item \hyperref[algebraicgroups-section-phantom]{Linear algebraic groups}
\item \hyperref[reductiveforms-section-phantom]{Forms and covers of reductive groups, and the $L$-group}
\item \hyperref[vermamodules-section-phantom]{Verma modules}
\item \hyperref[representations-local-section-phantom]{Representations of reductive groups over local fields}
%\item \hyperref[gKmodules-section-phantom]{$(\mathfrak g, K)$-modules}
%\item \hyperref[asymptotics-section-phantom]{Asymptotics and the Langlands classification}
\item \hyperref[plancherel-section-phantom]{Plancherel formula: reduction to discrete spectra}
\item \hyperref[discreteseries-section-phantom]{Construction of discrete series}
\item \hyperref[galoiscohomology-section-phantom]{Galois cohomology of linear algebraic groups}
\item \hyperref[automorphicspace-section-phantom]{The automorphic space}
%\item \hyperref[harmonicanalysis-section-phantom]{Harmonic analysis over local fields}
%\item \hyperref[automorphicforms-section-phantom]{Automorphic forms}
%\item \hyperref[periods-section-phantom]{Periods, theta correspondence, related methods}
%\item \hyperref[traceformulalocal-section-phantom]{The trace formula: local aspects}
%\item \hyperref[traceformulaglobal-section-phantom]{The trace formula: global aspects}
%\item \hyperref[arithmetic-section-phantom]{Arithmetic, reciprocity, Shimura varieties}
%\item \hyperref[geometric-section-phantom]{Geometric aspects}
\item \hyperref[fdl-section-phantom]{GNU Free Documentation License}
\item \hyperref[index-section-phantom]{Auto Generated Index}
\end{enumerate}
\end{multicols}





\bibliography{my}
\bibliographystyle{amsalpha}

\end{document}
