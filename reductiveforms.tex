\IfFileExists{stacks-project.cls}{%
\documentclass{stacks-project}
}{%
\documentclass{amsart}
}

% The following AMS packages are automatically loaded with
% the amsart documentclass:
%\usepackage{amsmath}
%\usepackage{amssymb}
%\usepackage{amsthm}

\usepackage{amssymb}

% For dealing with references we use the comment environment
\usepackage{verbatim}
\newenvironment{reference}{\comment}{\endcomment}
%\newenvironment{reference}{}{}
\newenvironment{slogan}{\comment}{\endcomment}
\newenvironment{history}{\comment}{\endcomment}

% For commutative diagrams you can use
% \usepackage{amscd}
\usepackage[all]{xy}

% We use 2cell for 2-commutative diagrams.
\xyoption{2cell}
\UseAllTwocells

% To put source file link in headers.
% Change "template.tex" to "this_filename.tex"
% \usepackage{fancyhdr}
% \pagestyle{fancy}
% \lhead{}
% \chead{}
% \rhead{Source file: \url{template.tex}}
% \lfoot{}
% \cfoot{\thepage}
% \rfoot{}
% \renewcommand{\headrulewidth}{0pt}
% \renewcommand{\footrulewidth}{0pt}
% \renewcommand{\headheight}{12pt}

\usepackage{multicol}

% For cross-file-references
\usepackage{xr-hyper}

% Package for hypertext links:
\usepackage{hyperref}

% For any local file, say "hello.tex" you want to link to please
% use \externaldocument[hello-]{hello}
\externaldocument[introduction-]{introduction}
\externaldocument[representationtheory-]{representationtheory}
\externaldocument[representations-compact-]{representations-compact}
\externaldocument[liegroups-general-]{liegroups-general}
\externaldocument[liestructure-]{liestructure} 
\externaldocument[algebraicgroups-]{algebraicgroups}
\externaldocument[reductiveforms-]{reductiveforms}
\externaldocument[vermamodules-]{vermamodules}
\externaldocument[representations-local-]{representations-local}
%\externaldocument[gKmodules-]{gKmodules}
%\externaldocument[asymptotics-]{asymptotics}
\externaldocument[plancherel-]{plancherel}
\externaldocument[discreteseries-]{discreteseries}
\externaldocument[galoiscohomology-]{galoiscohomology}
\externaldocument[automorphicspace-]{automorphicspace}
%\externaldocument[harmonicanalysis-]{harmonicanalysis} 
%\externaldocument[automorphicforms-]{automorphicforms}
%\externaldocument[periods-]{periods}
%\externaldocument[traceformulalocal-]{traceformulalocal}
%\externaldocument[traceformulaglobal-]{traceformulaglobal}
%\externaldocument[arithmetic-]{arithmetic}
%\externaldocument[geometric-]{geometric}
\externaldocument[fdl-]{fdl}
\externaldocument[index-]{index}

% Theorem environments.
%
\theoremstyle{plain}
\newtheorem{theorem}[subsection]{Theorem}
\newtheorem{proposition}[subsection]{Proposition}
\newtheorem{lemma}[subsection]{Lemma}

\theoremstyle{definition}
\newtheorem{definition}[subsection]{Definition}
\newtheorem{example}[subsection]{Example}
\newtheorem{exercise}[subsection]{Exercise}
\newtheorem{situation}[subsection]{Situation}

\theoremstyle{remark}
\newtheorem{remark}[subsection]{Remark}
\newtheorem{remarks}[subsection]{Remarks}

\numberwithin{equation}{subsection}

% Macros
%
\def\lim{\mathop{\rm lim}\nolimits}
\def\colim{\mathop{\rm colim}\nolimits}
\def\Spec{\mathop{\rm Spec}}
\def\Hom{\mathop{\rm Hom}\nolimits}
\def\SheafHom{\mathop{\mathcal{H}\!{\it om}}\nolimits}
\def\SheafExt{\mathop{\mathcal{E}\!{\it xt}}\nolimits}
\def\Sch{\textit{Sch}}
\def\Mor{\mathop{\rm Mor}\nolimits}
\def\Ob{\mathop{\rm Ob}\nolimits}
\def\Sh{\mathop{\textit{Sh}}\nolimits}
\def\NL{\mathop{N\!L}\nolimits}
\def\proetale{{pro\text{-}\acute{e}tale}}
\def\etale{{\acute{e}tale}}
\def\QCoh{\textit{QCoh}}
\def\Ker{\text{Ker}}
\def\Im{\text{Im}}
\def\Coker{\text{Coker}}
\def\Coim{\text{Coim}}

\def\eqref #1{(\ref{#1})}
\newcommand{\sslash}{\mathbin{/\mkern-6mu/}}


% OK, start here.
%
\begin{document}

\title{Forms and covers of reductive groups, and the $L$-group}


\maketitle

\phantomsection
\label{section-phantom}

\tableofcontents



\section{Classification of reductive groups over an algebraically closed field}
\label{section-classification-reductive}


For this section we assume $k$ to be algebraically closed. [For now, this section is missing a lot of results, as one needs to establish the analogs of results that we proved for semisimple Lie algebras in characteristic zero, for reductive groups.]

We saw in Theorem \ref{liestructure-theorem-diagonalizable-equivalence} a simple combinatorial description for diagonalizable groups, and we would like to have a similar description for more general reductive groups. This is not possible in the sense of getting an equivalence of categories\footnote{There is a good reason for it: To get an equivalence of categories one must consider the category of all $G$-representations, cf.\ Tannaka-Krein duality. For diagonalizable groups this category is described easily in terms of combinatorial data, this is no longer the case for other groups.}, but at least we can fully describe the isomorphism classes this way.


Let $G$ be reductive (over $k$), and let $T$ be a maximal torus in $G$. Let $X^\bullet(T)$, $X_\bullet(T)$ be the character and cocharacter groups of $T$. The adjoint action of $T$ on $\mathfrak g$ is semisimple, and we have a decomposition 
$$\mathfrak g = \mathfrak g_0 \oplus \bigoplus_{\alpha\in\Phi} \mathfrak g_\alpha,$$
where the $\alpha$'s, here, are eigencharacters $\alpha: T\to \mathbf G_m$. 

By Proposition \ref{liestructure-proposition-Cartan-reductive}, $\mathfrak g_0$ is just the Lie algebra of $T$, however, \emph{to accommodate the case of positive characteristic}, we will not be using $\mathfrak t$ to denote this Lie algebra, but the real vector space
$$ \mathfrak t = X_\bullet(T)\otimes \mathbb R$$
(and, similarly, $\mathfrak t^*= X^\bullet(T)\otimes \mathbb R$). For real algebraic groups, this $\mathfrak t^*$ can be identified with the dual of the Lie algebra,  by the map that assigns to any character $\chi$ its differential $d\chi$ at the identity. 

\begin{definition}[Root datum] 
\label{definition-root-datum}
 A {\it root datum} is a quadruple $(X,\Phi, \check X,\check\Phi)$, where $X, \check X$ are two lattices (finitely generated, torsion-free abelian groups) in duality, $\Phi\subset X$ and $\check\Phi\subset\check X$ are finite subsets, such that there exists a bijection $\Phi \ni \alpha\leftrightarrow \check\alpha\in\check\Phi$, satisfying:
\begin{enumerate}
\item $\left< \alpha,\check\alpha\right>=2$;
\item the endomorphisms of $X,\check X$ defined by $w_\alpha(x):= x-\left< x, \check\alpha\right> \alpha$, $w_{\check\alpha}(\check x)= \check x - \left< \alpha,\check x\right> \check\alpha$ preserve $\Phi$ and $\check\Phi$.
\end{enumerate}

A {\it based root datum} is a root datum as above, together with a choice of positive roots $\Phi^+\subset \Phi$ (in the sense of Definition \ref{liestructure-definition-based-root-system}).
\end{definition}

\begin{remarks}
\label{remarks-root-datum}
 \begin{enumerate}
  \item The last axiom is equivalent to: the endomorphisms $w_\alpha$ preserve $\Phi$ and generate a finite group (the Weyl group $W$).
  \item The condition $\left< \alpha,\check\alpha\right>=2$ characterizes the bijection $\Phi\leftrightarrow \check\Phi$, so it need not be part of the data.
  \item The $\mathbb R$-span of $\Phi$ in $X^*\otimes \mathbb R$, together with $\Phi$ and the Weyl group of automorphisms generated by the $w_\alpha$'s, forms a root system, as can be easily verified; hence the definition of based root datum in terms of based root systems.
 \end{enumerate}

\end{remarks}


We now define the appropriate notion of morphisms between root data.

\begin{definition}
\label{definition-isogeny-root-data}
 An {\it isogeny} of root data $(X,\Phi, \check X,\check\Phi) \to (X',\Phi', \check X',\check\Phi')$ is a homomorphism $f:X\to X'$ with the following properties:
 \begin{enumerate}
  \item $f$ is injective, and with finite cokernel --- hence, so is its adjoint $f^*:\check X'\to \check X$;
  \item $f$ and $f^*$ induce bijections between the subsets of roots and coroots, respectively.
 \end{enumerate}
 \end{definition}





\begin{proposition}
\label{proposition-simple-reflections}
 Given a connected reductive group $G$ and a maximal torus $T$ with set of roots $\Phi\subset X^*(T)$, and given $\alpha\in \Phi$, consider the subtorus $T_\alpha=\text{ker}(\alpha)^\circ\subset T$. Let $L_\alpha$ be the centralizer of $T_\alpha$ (which is connected by Proposition \ref{liestructure-proposition-centralizers-tori-connected}, and $L_\alpha'$ its derived subgroup. Then $L_\alpha'$ is isomorphic to $\text{SL}_2$ or $\text{PGL}_2$, and therefore there is a unique cocharacter $\check\alpha:\mathbf G_m\to T\cap L_{\alpha}'$ with $\left< \alpha, \check\alpha\right>=2$. If $w_\alpha$ is the nontrivial element of the Weyl group of $T$ inside of $L_\alpha'$, then the elements $w_\alpha$ generate the full Weyl group of $T$ in $G$, $W = \mathcal N_G(T)/T$.
\end{proposition}

\begin{proof}
 Omitted.
\end{proof}

\begin{definition}
 \label{definition-coroots-group}
In the notation of Proposition \ref{proposition-simple-reflections}, the elements $\check\alpha\in X_*(T)$ associated to the roots $\alpha\in X^*(T)$ are the {\it coroots} of the torus $T$ in $G$.
\end{definition}


\begin{definition}
\label{definition-central-isogeny}
 Let $(G,T)$, $(G',T')$ be two pairs consisting of a connected reductive group and a maximal torus. A {\it central isogeny} $(G',T')\to (G,T)$ is a morphism $G'\to G$ sending $T'\to T$, surjective, with finite kernel, and inducing isomorphisms between the root spaces of $T'$ and $T$ in $\mathfrak g'$, resp.\ $\mathfrak g$. A central isogeny $G'\to G$ is a morphism  which induces a central isogeny $(G',T')\to (G,T)$ for some maximal tori $T', T$ (or, equivalently, for any maximal torus $T'$, and $T$ the image of $T'$). 
\end{definition}

\begin{remark}
 \label{remark-on-central-isogenies}
A surjective morphism $(G',T')\to (G,T)$ with finite kernel gives rise to an injective map $X^*(T)\to X^*(T')$ with finite cokernel, and a bijection between the roots of $T$ on $G$ and the roots of $T'$ on $G'$. The requirement that the map induce an isomorphism between root spaces: $\mathfrak g_\alpha'\to \mathfrak g_\alpha$ is automatically satisfied in characteristic zero, but not in positive characteristic. For example, if $G$ is defined over a finite field with $q$ elements, the Frobenius morphism $F_q$ (see the proof of Theorem \ref{liestructure-theorem-maximal-tori-exist}) restricts to the Frobenius morphism on the subgroups $\mathfrak g_\alpha\simeq \mathbf G_a$, which is not an isomorphism.
\end{remark}


Now we define some categories of reductive groups with extra data that will be used in the classification.

\begin{definition}
 \label{definition-reductive-categories}
We let $Red$ be the category whose objects are reductive groups $G$ over $k$, and whose morphisms are central isogenies $G'\to G$. We let $Red_T$ be the category whose objects are pairs $(G\supset T)$ of reductive groups with maximal tori, and whose morphisms are central isogenies of pairs $(G',T')\to (G,T)$. We let $Red_{B,T}$ be the category whose objects are triples $(G\supset B\supset T)$ of reductive groups with Borel subgroups and maximal tori thereof, and whose morphisms are central isogenies of pairs $(G',T')\to (G,T)$ sending $B'$ to $B$.

We let $RD$ be the category whose objects are root data $(X,\Phi, \check X,\check\Phi)$ and whose morphisms are central isogenies. We let $RD^+$ be the category whose objects are root data, together with a choice $\Phi^+\subset \Phi$ of positive roots.
\end{definition}


\begin{theorem}[Classification over the algebraic closure]
\label{theorem-classification-reductive}
 Assume $k$ algebraically closed. Given a connected reductive group $G$ and a maximal torus $T$, the quadruple $\Psi(G,T)=(X^*(T), \Phi, X_*(T), \check\Phi)$, where $\Phi,\check\Phi$ denote, respectively, the sets of roots and coroots of $T$ in $G$, is a root datum.
 
 The assignment $(G,T)\mapsto \Psi(G,T)$ is a functor $Red \to RD^{\text{op}}$, in the notation of Definition \ref{definition-reductive-categories}.
 
 This functor is a bijection on isomorphism classes of objects, and every morphism $\Psi(G,T)\to \Psi(G',T')$ in $RD$ lifts to a morphism (i.e., central isogeny) $(G',T')\to (G,T)$, uniquely up to precomposing with conjugation by elements of $T'$, or post-composing with conjugation by elements of $T$.
\end{theorem}

\begin{proof}
 Omitted. See \cite{Springer-Corvallis} for more details and references.
\end{proof}

We will now see two variants of this theorem: One, where we choose more data in order to rigidify the functor, and another, where we make no choices (i.e., we do not choose tori).


If, in the context of Theorem \ref{theorem-classification-reductive}, we choose a Borel subgroup $B\supset T$, it gives rise to a based root datum, with $\Phi^+$ being the set of roots appearing in the Lie algebra of $B$. We will generally denote by $\Delta$ the subset of simple roots.

\begin{definition}
 \label{definition-pinning}
A {\it pinning} of a triple $(G, B, T)$ consisting of a connected reductive group $G$, a Borel subgroup,  and a maximal torus $T$ is a choice of isomorphisms $p_\alpha:\mathfrak g_\alpha \simeq \mathbf G_a$, for every simple root $\alpha\in \Delta$.

We let $Red_{\text{PIN}}$ denote the category whose objects are pinned reductive groups, and whose morphisms are central isogenies preserving the Borel, the torus, and the pinning.
\end{definition}

Pinnings allow us to rigidify the morphisms lifted from morphisms of (based) root data:

\begin{theorem}
\label{theorem-classification-with-pinning}
The assignment $(G, B, T, P) \to \Psi^+(G,B,T)$, where $\Psi^+(G,B,T)$ denotes the based root datum consisting of $\Psi(G,T)$ (notation as in Theorem \ref{theorem-classification-reductive}) with $\Phi^+$ the positive roots associated to $B$, is an equivalence of categories: $Red_{\text{PIN}}\to RD^+$.
\end{theorem}

\begin{proof}
 Omitted. See, again, \cite{Springer-Corvallis}. 
\end{proof}

Finally, a version of the classification for morphisms $G'\to G$, without any choices: Recall that to any reductive group $G$, we have associated its (universal) Cartan $\mathbb A^G$, endowed with sets of roots and positive roots $\Phi^+\subset \Phi\subset X^*(\mathbb A^G)$. (Recall that in this section we are assuming the field to be algebraically closed.) We will denote by $\Psi^+(\mathbb A^G)$ this based root system.

\begin{lemma}
 \label{lemma-Cartan-functorial}
The association $G\to \mathbb A^G$ is functorial in the category $Red$ of reductive groups with central isogenies.
\end{lemma}

\begin{proof}
 If $f:G\to G'$ is a central isogeny, it sends a Borel subgroup $B\subset G$ onto a Borel subgroup $B'\to G'$, inducing morphisms of their reductive quotients: $\mathbb A^G=B/N\to \mathbb A^{G'}=B'/N'$. If we choose another Borel subgroup $B_1\subset G$, there is a $g\in G$ with $g B g^{-1} = B_1$, hence $f(g) B' f(g)^{-1} = B_1':=f(B_1)$, and the morphism $\mathbb A^G\to \mathbb A^{G'}$ induced from $B_1\to B_1'$ is the same as before, given how we have identified $\mathbb A^G$ as the quotient of \emph{any} Borel subgroup.
\end{proof}


\begin{theorem}
\label{theorem-classification-with-universal-Cartan}
The assignment $G \to \Psi^+(\mathbb A^G)$ is a functor $Red\to (RD^+)^{\text{op}}$, in the notation of Definition \ref{definition-reductive-categories}. It induces a bijection on isomorphism classes of objects, and for any morphism $\Psi^+(\mathbb A^{G}) \to \Psi^+(\mathbb A^{G'})$ there is a morphism $G'\to G$, unique up to inner automorphisms.
\end{theorem}

\begin{proof}
 This follows from Theorem \ref{theorem-classification-reductive}, and the conjugacy of Cartan subgroups.
\end{proof}





\subsection{Simply connected and adjoint groups}
\label{subsection-simply-connected-adjoint}

\begin{definition}
 \label{definition-sc-adjoint-rootdatum}
Let $(X,\Phi,\check X,\check\Phi)$ be a root datum, let $R\subset X$, $\check R\subset \check X$ be the subgroups spanned by the roots, resp.\ coroots, and let $P = \check R^*$, $\check P = R^*$ the dual lattices. 

The root datum is called {\it semisimple} if $\mathcal R$ is of finite index in $X$. 

Assume this to be the case, so that we have containments with finite quotients $P\supset X\supset R$ and $\check P\supset\check X\supset \check R$. We say that the root datum  is {\it simply-connected} if $X=\mathcal P$, equivalently $\check X=\check R$, and {\it adjoint} if $X=\mathcal R$, equivalently $\check X =\check P$. 
\end{definition}

\begin{definition}
 \label{definition-sc-adjoint-group}
A reductive group is {\it adjoint} if it has trivial center, and {\it simply connected} if it admits no (connected) central isogenies.
\end{definition}

The relationship between this notion of being simply connected, and the algebrogeometric one, will become clear in Proposition  \ref{proposition-sc-adjoint-group} and Remark \ref{remark-sc}.




\begin{proposition}
 \label{proposition-sc-adjoint-group}
Let $(G,T)$ be a connected reductive group with a maximal torus over a field $k$, and let $\Psi(G,T)$ be the associated root datum.
\begin{enumerate}
\item $G$ is semisimple iff $\Psi(G,T)$ is semisimple.
\item $G$ is adjoint iff $\Psi(G,T)$ is adjoint.
\item $G$ is simply connected (in the sense of Definition \ref{definition-sc-adjoint-group} iff $\Psi(G,T)$ is simply connected. 
\item In characteristic zero, $G$ is simply connected as a scheme (or, equivalently, $G(\mathbb C)$ is simply connected as a topological space when $k=\mathbb C$) iff $\Psi(G,T)$ is simply connected.
\end{enumerate}

\end{proposition}

\begin{proof}
\begin{enumerate}
 \item A reductive group is semisimple iff its center is finite. The center belongs to $T$, and coincides with the common kernel of all roots. In other words, in terms of the equivalence of diagonalizable groups and finitely generated abelian groups (Theorem \ref{liestructure-theorem-diagonalizable-equivalence}), the character group of the center is the cokernel of the inclusion $R\to X$. Hence, it is finite iff $R$ is of finite index in $X$.
 
 (Notice that the center is not necessarily reduced; for example, the center of $\text{SL}_p$ in characteristic $p$ is the non-reduced group scheme $\mu_p$ of $p$-th roots of unity.)
 
 \item Continuing along the same lines, the center is trivial iff $R=X$.
 
 \item A central isogeny $G'\to G$ which is not an isomorphism, mapping some maximal tori $T'\to T$,  induces an isogeny $\Psi(G,T)\to \Psi(G',T')$, which is completely determined by the map of cocharacter groups $X_*(T')\to X_*(T)$. This map is injective, of finite cokernel, and has to preserve coroot lattices, so if $X_*(T)$ is equal to the coroot lattice, the map is an isomorphism.
 
 \item By \cite[Theorem 1]{Brion-Szamuely}, if $p$ is the characteristic exponent of the field, every prime-to-$p$ \'etale Galois cover $G'\to G$ is a central isogeny. In particular, in characteristic zero, $G$ is simply connected in the sense of Definition \ref{definition-sc-adjoint-group} iff it is simply connected in the sense of \'etale topology. Moreover, if $k=\mathbb C$, the \'etale fundamental group is the profinite completion of the topological fundamental group of $G(\mathbb C)$. 
 
\end{enumerate}

\end{proof}


\begin{remark}
\label{remark-sc} 
In positive characteristic, the \'etale fundamental group is always infinite, for a smooth affine scheme $X= \text{Spec}(R)$ of positive dimension. For example, we have the Artin--Schreier $\mathbb Z/p$-covers, $X'=\text{Spec} R[y]/(y^p-y-f)$, if $f\in R$ is chosen appropriately.
\end{remark}




\begin{multicols}{2}[\section{Other chapters}]
\noindent
\begin{enumerate}
\item \hyperref[introduction-section-phantom]{Introduction}
\item \hyperref[representationtheory-section-phantom]{Basic Representation Theory}
\item \hyperref[representations-compact-section-phantom]{Representations of compact groups}
\item \hyperref[liegroups-general-section-phantom]{Lie groups and Lie algebras: general properties}
\item \hyperref[liestructure-section-phantom]{Structure of finite-dimensional Lie algebras}
\item \hyperref[algebraicgroups-section-phantom]{Linear algebraic groups}
\item \hyperref[reductiveforms-section-phantom]{Forms and covers of reductive groups, and the $L$-group}
\item \hyperref[vermamodules-section-phantom]{Verma modules}
\item \hyperref[representations-local-section-phantom]{Representations of reductive groups over local fields}
%\item \hyperref[gKmodules-section-phantom]{$(\mathfrak g, K)$-modules}
%\item \hyperref[asymptotics-section-phantom]{Asymptotics and the Langlands classification}
\item \hyperref[plancherel-section-phantom]{Plancherel formula: reduction to discrete spectra}
\item \hyperref[discreteseries-section-phantom]{Construction of discrete series}
\item \hyperref[galoiscohomology-section-phantom]{Galois cohomology of linear algebraic groups}
\item \hyperref[automorphicspace-section-phantom]{The automorphic space}
%\item \hyperref[harmonicanalysis-section-phantom]{Harmonic analysis over local fields}
%\item \hyperref[automorphicforms-section-phantom]{Automorphic forms}
%\item \hyperref[periods-section-phantom]{Periods, theta correspondence, related methods}
%\item \hyperref[traceformulalocal-section-phantom]{The trace formula: local aspects}
%\item \hyperref[traceformulaglobal-section-phantom]{The trace formula: global aspects}
%\item \hyperref[arithmetic-section-phantom]{Arithmetic, reciprocity, Shimura varieties}
%\item \hyperref[geometric-section-phantom]{Geometric aspects}
\item \hyperref[fdl-section-phantom]{GNU Free Documentation License}
\item \hyperref[index-section-phantom]{Auto Generated Index}
\end{enumerate}
\end{multicols}





\bibliography{my}
\bibliographystyle{amsalpha}

\end{document}



