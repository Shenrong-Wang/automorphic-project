\IfFileExists{stacks-project.cls}{%
\documentclass{stacks-project}
}{%
\documentclass{amsart}
}

% The following AMS packages are automatically loaded with
% the amsart documentclass:
%\usepackage{amsmath}
%\usepackage{amssymb}
%\usepackage{amsthm}

\usepackage{amssymb}

% For dealing with references we use the comment environment
\usepackage{verbatim}
\newenvironment{reference}{\comment}{\endcomment}
%\newenvironment{reference}{}{}
\newenvironment{slogan}{\comment}{\endcomment}
\newenvironment{history}{\comment}{\endcomment}

% For commutative diagrams you can use
% \usepackage{amscd}
\usepackage[all]{xy}

% We use 2cell for 2-commutative diagrams.
\xyoption{2cell}
\UseAllTwocells

% To put source file link in headers.
% Change "template.tex" to "this_filename.tex"
% \usepackage{fancyhdr}
% \pagestyle{fancy}
% \lhead{}
% \chead{}
% \rhead{Source file: \url{template.tex}}
% \lfoot{}
% \cfoot{\thepage}
% \rfoot{}
% \renewcommand{\headrulewidth}{0pt}
% \renewcommand{\footrulewidth}{0pt}
% \renewcommand{\headheight}{12pt}

\usepackage{multicol}

% For cross-file-references
\usepackage{xr-hyper}

% Package for hypertext links:
\usepackage{hyperref}

% For any local file, say "hello.tex" you want to link to please
% use \externaldocument[hello-]{hello}
\externaldocument[introduction-]{introduction}
\externaldocument[representationtheory-]{representationtheory}
\externaldocument[representations-compact-]{representations-compact}
\externaldocument[liegroups-general-]{liegroups-general}
\externaldocument[liestructure-]{liestructure} 
\externaldocument[algebraicgroups-]{algebraicgroups}
\externaldocument[reductiveforms-]{reductiveforms}
\externaldocument[vermamodules-]{vermamodules}
\externaldocument[representations-local-]{representations-local}
%\externaldocument[gKmodules-]{gKmodules}
%\externaldocument[asymptotics-]{asymptotics}
\externaldocument[plancherel-]{plancherel}
\externaldocument[discreteseries-]{discreteseries}
\externaldocument[galoiscohomology-]{galoiscohomology}
\externaldocument[automorphicspace-]{automorphicspace}
%\externaldocument[harmonicanalysis-]{harmonicanalysis} 
%\externaldocument[automorphicforms-]{automorphicforms}
%\externaldocument[periods-]{periods}
%\externaldocument[traceformulalocal-]{traceformulalocal}
%\externaldocument[traceformulaglobal-]{traceformulaglobal}
%\externaldocument[arithmetic-]{arithmetic}
%\externaldocument[geometric-]{geometric}
\externaldocument[fdl-]{fdl}
\externaldocument[index-]{index}

% Theorem environments.
%
\theoremstyle{plain}
\newtheorem{theorem}[subsection]{Theorem}
\newtheorem{proposition}[subsection]{Proposition}
\newtheorem{lemma}[subsection]{Lemma}

\theoremstyle{definition}
\newtheorem{definition}[subsection]{Definition}
\newtheorem{example}[subsection]{Example}
\newtheorem{exercise}[subsection]{Exercise}
\newtheorem{situation}[subsection]{Situation}

\theoremstyle{remark}
\newtheorem{remark}[subsection]{Remark}
\newtheorem{remarks}[subsection]{Remarks}

\numberwithin{equation}{subsection}

% Macros
%
\def\lim{\mathop{\rm lim}\nolimits}
\def\colim{\mathop{\rm colim}\nolimits}
\def\Spec{\mathop{\rm Spec}}
\def\Hom{\mathop{\rm Hom}\nolimits}
\def\SheafHom{\mathop{\mathcal{H}\!{\it om}}\nolimits}
\def\SheafExt{\mathop{\mathcal{E}\!{\it xt}}\nolimits}
\def\Sch{\textit{Sch}}
\def\Mor{\mathop{\rm Mor}\nolimits}
\def\Ob{\mathop{\rm Ob}\nolimits}
\def\Sh{\mathop{\textit{Sh}}\nolimits}
\def\NL{\mathop{N\!L}\nolimits}
\def\proetale{{pro\text{-}\acute{e}tale}}
\def\etale{{\acute{e}tale}}
\def\QCoh{\textit{QCoh}}
\def\Ker{\text{Ker}}
\def\Im{\text{Im}}
\def\Coker{\text{Coker}}
\def\Coim{\text{Coim}}

\def\eqref #1{(\ref{#1})}
\newcommand{\sslash}{\mathbin{/\mkern-6mu/}}


% OK, start here.
%
\begin{document}

\title{Geometric aspects}


\maketitle

\phantomsection
\label{section-phantom}

\begin{verbatim}
Copyright (C) 2005 -- 2016 Johan de Jong
Permission is granted to copy, distribute and/or modify this
document under the terms of the GNU Free Documentation License,
Version 1.2 or any later version published by the Free Software
Foundation; with no Invariant Sections, no Front-Cover Texts,
and no Back-Cover Texts. A copy of the license is included in
the section entitled "GNU Free Documentation License".
\end{verbatim}

\tableofcontents


\section{Overview}
\label{section-overview}

\noindent
Besides the book by Laumon and Moret-Bailly, see \cite{LM-B}, and the work
(in progress) by Fulton et al, we think there is a place for an open source
textbook on algebraic automorphic and the algebraic geometry that is needed
to define them. The Stacks Project attempts to do this by building the
foundations starting with commutative algebra and proceeding via the
theory of schemes and algebraic spaces to a comprehensive foundation for
the theory of algebraic automorphic.

\medskip\noindent
We expect this material to be read online as a key feature are the hyperlinks
giving quick access to internal references spread over many different pages.
If you use an embedded pdf or dvi viewer in your browser, the cross file
links should work.

\medskip\noindent
This project is a collaborative effort and we encourage you to help out.
Please email any typos or errors you find while reading or
any suggestions, additional material, or examples you have to
\href{mailto:automorphic.project@gmail.com}{automorphic.project@gmail.com}.
You can download a tarball containing all source files, extract,
run make, and use a dvi or pdf viewer locally. Please feel free to
edit the LaTeX files and email your improvements.


\section{Attribution}
\label{section-attribution}

\noindent
The scope of this work is such that it is a daunting task to attribute
correctly and succinctly all of those mathematicians whose work has led
to the development of the theory we try to explain here. We hope eventually
to generate enough community interest to find contributors willing to write
sections with historical remarks for each and every chapter.

\medskip\noindent
Those who contributed to this work are listed on the title page of the book
version of this work and
\href{http://automorphic.newark.rutgers.edu/tex/CONTRIBUTORS}{online}.
Here we would like to name a selection of major contributions:
\begin{enumerate}
\item Jarod Alper wrote
\hyperref[guide-section-phantom]{Guide to Literature}.
\item Bhargav Bhatt wrote the initial version of
\hyperref[etale-section-phantom]{\'Etale Morphisms of Schemes}.
\item Bhargav Bhatt wrote the initial version of
More on Algebra, Section \ref{more-algebra-section-formal-glueing}.
\item Kiran Kedlaya contributed the initial writeup of
Descent, Section \ref{descent-section-descent-universally-injective}.
\item The initial versions of
\begin{enumerate}
\item Algebra, Section \ref{algebra-section-oka-families},
\item Injectives, Section \ref{injectives-section-baer}, and
\item the chapter \hyperref[fields-section-phantom]{Fields}
\end{enumerate}
are from
\href{http://people.fas.harvard.edu/~amathew/cr.html}{The CRing Project},
courtesy of Akhil Mathew et al.
\item Alex Perry wrote the material on projective modules,
Mittag-Leffler modules, including the proof of
Algebra, Theorem \ref{algebra-theorem-ffdescent-projectivity}.
\item Alex Perry wrote
\hyperref[formal-defos-section-phantom]{Formal Deformation Theory}.
\item Thibaut Pugin, Zachary Maddock and Min Lee took course notes
which formed the basis for
\hyperref[etale-cohomology-section-phantom]{\'Etale Cohomology}.
\item David Rydh has contributed many helpful comments, pointed out several
mistakes, helped out in an essential way with the material on residual gerbes,
and was the originator for the material in
More on Groupoids in Spaces, Sections
\ref{spaces-more-groupoids-section-finite} and
\ref{spaces-more-groupoids-section-etale-localize}.
\item Burt Totaro contributed Examples, Sections
\ref{examples-section-non-descending-property-projective},
\ref{examples-section-non-effective-descent-projective},
and
Properties of Stacks, Section \ref{automorphic-properties-section-dimension}.
\item The material in the chapter
\hyperref[proetale-section-phantom]{Pro-\'etale Cohomology}
is taken from a paper by Bhargav Bhatt and Peter Scholze.
\item Bhargav Bhatt contributed Examples, Sections
\ref{examples-section-non-algebraic-hom-stack} and
\ref{examples-section-flat-not-colimit-flat-finitely-presented}.
\item Ofer Gabber found mistakes, contributed corrections and he
contributed
Varieties, Lemma \ref{varieties-lemma-image-connected-component},
Formal Spaces, Lemma \ref{formal-spaces-lemma-completion-countably-indexed},
the material in
More on Groupoids, Section \ref{more-groupoids-section-ind-quasi-affine},
the main result of
Properties of Spaces, Section \ref{spaces-properties-section-fpqc},
and the proof of
More on Flatness, Proposition
\ref{flat-proposition-finite-type-injective-into-flat-mod-m}.
\item J\'anos Koll\'ar contributed
Algebra, Lemma \ref{algebra-lemma-hart-serre-loc-thm} and
Dualizing Complexes, Proposition \ref{dualizing-proposition-kollar}.
\item Kiran Kedlaya wrote the initial version of
More on Algebra, Section \ref{more-algebra-section-beauville-laszlo}.
\item Matthew Emerton, Toby Gee, and Brandon Levin contributed
some results on thickenings, in particular
More on Morphisms of Stacks, Lemmas
\ref{automorphic-more-morphisms-lemma-reduced-diagonal},
\ref{automorphic-more-morphisms-lemma-thickening-diagonals}, and
\ref{automorphic-more-morphisms-lemma-thickening-properties}.
\item Lena Min Ji wrote the initial version of
More on Algebra, Section \ref{more-algebra-section-principal-radical-ideals}.
\end{enumerate}



\begin{multicols}{2}[\section{Other chapters}]
\noindent
\begin{enumerate}
\item \hyperref[introduction-section-phantom]{Introduction}
\item \hyperref[representationtheory-section-phantom]{Basic Representation Theory}
\item \hyperref[representations-compact-section-phantom]{Representations of compact groups}
\item \hyperref[liegroups-general-section-phantom]{Lie groups and Lie algebras: general properties}
\item \hyperref[liestructure-section-phantom]{Structure of finite-dimensional Lie algebras}
\item \hyperref[algebraicgroups-section-phantom]{Linear algebraic groups}
\item \hyperref[reductiveforms-section-phantom]{Forms and covers of reductive groups, and the $L$-group}
\item \hyperref[vermamodules-section-phantom]{Verma modules}
\item \hyperref[representations-local-section-phantom]{Representations of reductive groups over local fields}
%\item \hyperref[gKmodules-section-phantom]{$(\mathfrak g, K)$-modules}
%\item \hyperref[asymptotics-section-phantom]{Asymptotics and the Langlands classification}
\item \hyperref[plancherel-section-phantom]{Plancherel formula: reduction to discrete spectra}
\item \hyperref[discreteseries-section-phantom]{Construction of discrete series}
\item \hyperref[galoiscohomology-section-phantom]{Galois cohomology of linear algebraic groups}
\item \hyperref[automorphicspace-section-phantom]{The automorphic space}
%\item \hyperref[harmonicanalysis-section-phantom]{Harmonic analysis over local fields}
%\item \hyperref[automorphicforms-section-phantom]{Automorphic forms}
%\item \hyperref[periods-section-phantom]{Periods, theta correspondence, related methods}
%\item \hyperref[traceformulalocal-section-phantom]{The trace formula: local aspects}
%\item \hyperref[traceformulaglobal-section-phantom]{The trace formula: global aspects}
%\item \hyperref[arithmetic-section-phantom]{Arithmetic, reciprocity, Shimura varieties}
%\item \hyperref[geometric-section-phantom]{Geometric aspects}
\item \hyperref[fdl-section-phantom]{GNU Free Documentation License}
\item \hyperref[index-section-phantom]{Auto Generated Index}
\end{enumerate}
\end{multicols}




\bibliography{my}
\bibliographystyle{amsalpha}

\end{document}
