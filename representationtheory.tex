\IfFileExists{stacks-project.cls}{%
\documentclass{stacks-project}
}{%
\documentclass{amsart}
}

% The following AMS packages are automatically loaded with
% the amsart documentclass:
%\usepackage{amsmath}
%\usepackage{amssymb}
%\usepackage{amsthm}

\usepackage{amssymb}

% For dealing with references we use the comment environment
\usepackage{verbatim}
\newenvironment{reference}{\comment}{\endcomment}
%\newenvironment{reference}{}{}
\newenvironment{slogan}{\comment}{\endcomment}
\newenvironment{history}{\comment}{\endcomment}

% For commutative diagrams you can use
% \usepackage{amscd}
\usepackage[all]{xy}

% We use 2cell for 2-commutative diagrams.
\xyoption{2cell}
\UseAllTwocells

% To put source file link in headers.
% Change "template.tex" to "this_filename.tex"
% \usepackage{fancyhdr}
% \pagestyle{fancy}
% \lhead{}
% \chead{}
% \rhead{Source file: \url{template.tex}}
% \lfoot{}
% \cfoot{\thepage}
% \rfoot{}
% \renewcommand{\headrulewidth}{0pt}
% \renewcommand{\footrulewidth}{0pt}
% \renewcommand{\headheight}{12pt}

\usepackage{multicol}

% For cross-file-references
\usepackage{xr-hyper}

% Package for hypertext links:
\usepackage{hyperref}

% For any local file, say "hello.tex" you want to link to please
% use \externaldocument[hello-]{hello}
\externaldocument[introduction-]{introduction}
\externaldocument[representationtheory-]{representationtheory}
\externaldocument[representations-compact-]{representations-compact}
\externaldocument[liegroups-general-]{liegroups-general}
\externaldocument[liestructure-]{liestructure} 
\externaldocument[algebraicgroups-]{algebraicgroups}
\externaldocument[reductiveforms-]{reductiveforms}
\externaldocument[vermamodules-]{vermamodules}
\externaldocument[representations-local-]{representations-local}
%\externaldocument[gKmodules-]{gKmodules}
%\externaldocument[asymptotics-]{asymptotics}
\externaldocument[plancherel-]{plancherel}
\externaldocument[discreteseries-]{discreteseries}
\externaldocument[galoiscohomology-]{galoiscohomology}
\externaldocument[automorphicspace-]{automorphicspace}
%\externaldocument[harmonicanalysis-]{harmonicanalysis} 
%\externaldocument[automorphicforms-]{automorphicforms}
%\externaldocument[periods-]{periods}
%\externaldocument[traceformulalocal-]{traceformulalocal}
%\externaldocument[traceformulaglobal-]{traceformulaglobal}
%\externaldocument[arithmetic-]{arithmetic}
%\externaldocument[geometric-]{geometric}
\externaldocument[fdl-]{fdl}
\externaldocument[index-]{index}

% Theorem environments.
%
\theoremstyle{plain}
\newtheorem{theorem}[subsection]{Theorem}
\newtheorem{proposition}[subsection]{Proposition}
\newtheorem{lemma}[subsection]{Lemma}

\theoremstyle{definition}
\newtheorem{definition}[subsection]{Definition}
\newtheorem{example}[subsection]{Example}
\newtheorem{exercise}[subsection]{Exercise}
\newtheorem{situation}[subsection]{Situation}

\theoremstyle{remark}
\newtheorem{remark}[subsection]{Remark}
\newtheorem{remarks}[subsection]{Remarks}

\numberwithin{equation}{subsection}

% Macros
%
\def\lim{\mathop{\rm lim}\nolimits}
\def\colim{\mathop{\rm colim}\nolimits}
\def\Spec{\mathop{\rm Spec}}
\def\Hom{\mathop{\rm Hom}\nolimits}
\def\SheafHom{\mathop{\mathcal{H}\!{\it om}}\nolimits}
\def\SheafExt{\mathop{\mathcal{E}\!{\it xt}}\nolimits}
\def\Sch{\textit{Sch}}
\def\Mor{\mathop{\rm Mor}\nolimits}
\def\Ob{\mathop{\rm Ob}\nolimits}
\def\Sh{\mathop{\textit{Sh}}\nolimits}
\def\NL{\mathop{N\!L}\nolimits}
\def\proetale{{pro\text{-}\acute{e}tale}}
\def\etale{{\acute{e}tale}}
\def\QCoh{\textit{QCoh}}
\def\Ker{\text{Ker}}
\def\Im{\text{Im}}
\def\Coker{\text{Coker}}
\def\Coim{\text{Coim}}

\def\eqref #1{(\ref{#1})}
\newcommand{\sslash}{\mathbin{/\mkern-6mu/}}


% OK, start here.
%
\begin{document}

\title{Representation theory: general notions}


\maketitle

\phantomsection
\label{section-phantom}


\tableofcontents


\section{Conventions}
\label{section-conventions}

The definition of representations makes sense over an arbitrary field, but very soon we start working with measures, specializing to the complex numbers as the coefficient field.

\section{Representations}
\label{section-representations}

Let $G$ be a topological group. Topological vector spaces are taken over a 
topological field $k$ (which we fix). We denote by $\text{End}(V)$, 
$\text{Aut}(V)$ the sets of {\it continuous} endomorphisms, resp.\ automorphisms, of a topological vector space $V$.


\begin{definition}
\label{definition-representation}
A {\it representation} of $G$ is a pair $(\pi,V)$, where $V$ is a topological vector space $V$ over $k$, and $\pi$ is a 
homomorphism
$$\pi: G\to \text{Aut}(V),$$
with the property that the induced ``action'' map:
$$G\times V\to V,$$
$$(g,v)\mapsto \pi(g) v$$
is continuous.

Representations of $G$ on topological $k$-vector spaces form a category, with a morphism 
$$ (\pi_1, V_1)\to (\pi_2,V_2)$$
being a continuous linear map $V_1\to V_2$ which commutes with the action of $G$. 

A {\it subrepresentation} of $V$ is a {\it closed} subspace of $V$ which is stable under the action of $G$.

A representation is called {\it irreducible}, or {\it simple}, if it does not contain any non-zero, proper subrepresentations. 

A representation is called {\it semisimple} or {\it (totally) decomposable} if it is equal to the direct sum of irreducible subrepresentations. \footnote{For infinite-dimensional representations, other notions of decomposition, such as by {\it direct integrals}, are often more useful. They will be discussed later.} 
\end{definition}

\begin{example}
 \label{example-representation}
Given a space $X$ with a right action of a group $G$, if $F(X)$ denotes the space of $k$-valued functions on $X$, there is a natural representation of $G$ on $F(X)$, sometimes called the \emph{regular representation} of $G$ on $X$ (although the term is more standard for the action of $G$ on itself, see Definition \ref{definition-regular-representation}). It is given by 
$$ R(g)(f)(x) = f(xg).$$

If $G, X$ are endowed with topology, so that the map $X\times G\to X$ is continuous, and $X$ is locally compact and Hausdorff, the space $C(X)$ of continuous, complex-valued functions, topologized as the inductive limit over all compact $K\subset X$ of the Banach spaces $C(K)$, becomes a (topological) representation of $G$, and so does its dual $M_c(X)$ of compactly supported regular measures.

If $(X,\Omega, \mu)$ is a measure space with a right action by a discrete group $G$ that preserves the $\sigma$-algebra $\Omega$ and the measure $\mu$, the spaces $L^p(X,\mu)$ become topological representations of $G$. 

If $G$ is any topological group acting continuously on a locally compact Hausdorff topological space $X$ as before, and $\mu$ is a regular Borel measure preserved by the action of $G$, then $L^p(X,\mu)$ is a (continuous) representation of $G$ for $1\le p<\infty$. (Exercise!)        
\end{example}



\begin{remark}
\label{remark-continuity-operatortopology}
 We do not require the map $G\to \text{Aut}(V)$ to be continuous in any of the usual operator topologies (for example, the norm on bounded linear operators, when $V$ is a Banach space), because this would preclude some of the most natural representations. For example, the rotation representation of $G=$ the circle on $V=L^2(\mathbb R)$ is not a continuous map $G\to \text{Aut}(V)$ with respect to the Hilbert norm on bounded operators. This may seem troubling at first, but it will appear more natural when we talk about the action of group algebras, see Remark \ref{remark-continuity-algebras}.
\end{remark}




\section{Action of measures on the group}
\label{section-measures}

A basic principle in representation theory is that we should extend the action of the group to the action of a suitable algebra, because algebras have a lot more structure.


From now on we assume that the field $k$ of coefficients of the representation is the field $\mathbb C$ of complex numbers, and that the topological group $G$ is locally compact. 

Let $M(G)$ be the Banach space of finite, complex-valued measures on $G$. (``Measures'' will always mean Radon measures.) It is a Banach algebra under convolution. Convolution is, by definition, the push-forward of measures under the multiplication map $G\times G \to G$. We denote by $M_c(G)$ the subalgebra of compactly supported finite measures. If $dg$ is a left Haar measure on $G$, we will call a measure $\mu= f dg$ continuous, $L^1$, etc, if $f$ is a function with the same property.



If $(\pi, V)$ is a topological representation of $G$, we would like to extend the action of $G$ to an action of the algebra $M(G)$ of measures, or a subalgebra $A$ thereof, in such a way that the action of $g\in G$ will correspond to the action of the delta measure at $g$; we will keep using the notation $\pi$ for such an extension. 

\begin{example}
\label{example-groupalgebra}
The convolution algebra of $k$-valued measures of finite support on $G$ makes sense for an arbitrary commutative ring of coefficients $k$, and is called the {\it group algebra} $k[G]$. 

If $G$ is discrete, there is an (obvious) equivalence of categories between representations of $G$ on $k$-vector spaces (without topology, i.e., with the discrete topology) and $k[G]$-modules.
\end{example}



\begin{example}
\label{example-groupalgebra-Z}
In particular, for the group $G=\mathbb Z$, its group algebra is $k[T,T^{-1}]$, and its finitely generated representations (without topology) are classified by the structure theorem for finitely generated modules over principal ideal domains.
\end{example}



We assume throughout that the continuous dual $V^*$ of $V$ separates points. Then, the extension that we seek will be characterized by the property
\begin{equation}
\label{equation-action-measures}
\left<\pi(\mu)(v), v^*\right> = \int_G \left<\pi(g)v, v^*\right> \mu(g),
\end{equation}
for every $\mu\in A$, $v\in V$, $v^*\in V^*$. 

\begin{proposition}
\label{proposition-integral-lcs}
 Assume that $V$ is a locally convex topological vector space, and such that the closure of the convex hull of any compact set is compact. Then, for every $\mu\in M_c(G)$, $v\in V$ the vector $\pi(\mu)(v)$ characterized by \eqref{equation-action-measures} is defined. 
 
 The resulting map $M_c(G)\times V \to V$ is continuous, when $M_c(G)$ is topologized as the inductive limit, over compact subsets $K\subset G$, of the Banach spaces $M(K)$. 
 
 Moreover, the restriction of this map to any bounded subset of $M_c(G)$ is continuous with respect to the weak-* topology, that is, the inductive limit topology of the spaces $M(K)$ endowed with the weak-* topology as duals of the spaces $C(K)$.
\end{proposition}

Notice that by \cite[Theorem 3.20]{Rudin}, Fr\'echet spaces satisfy the conditions of the proposition.

\begin{proof}
By \cite[Theorem 3.27]{Rudin}, for a finite Borel measure $\mu$ on a compact Hausdorff space $K$, and a continuous function $f:K\to V$ to a topological vector space $V$ satisfying the conditions of the proposition, the integral $\int_K f \cdot \mu$, characterized by 
$$ \left< \int_K f \cdot \mu, v^*\right> = \int_K \left< f , v^*\right> \cdot \mu$$ for every $v^*\in V^*$, is defined. (The argument, in brief: First, reduce to the case of positive probability measures. Then, show that the subsets of the closed convex hull of $f(K)$ cut out by a finite number of conditions of the form $\left< v, v^*\right> = \int_K \left< f , v^*\right> \cdot \mu$ are nonempty. By compactness, it follows that their intersection is nonempty.) 

Here, we take $K \subset G$ to be the support of a given measure, and $f(g) = \pi(g)(v)$, and set $\pi(\mu)(v):= \int_{G} \pi(g)(v) \mu(g)$.

The same result ensures that, when $\mu$ is a positive probability measure, the integral $\int_{G} \pi(g)(v) \mu(g)$ belongs to the closure of the convex hull of the compact set $\pi(\text{supp} \mu)(v)$ (to be denoted $\overline{\text{co}} \left( \pi(\text{supp} \mu)(v) \right)$.

The continuity statement on (norm) bounded sets with the weak-* topology is stronger than the continuity statement in the norm topology, so it suffices to prove that. Fix a compact $K\subset G$, and let $M(K)_{\le 1}$ be the unit ball in $M(K)$. It is enough to prove that the preimage of a neighborhood $U$ of $0\in V$ contains a product $M \times U'$, where $M$ is a weak-* neighborhood of zero in $M(K)_{\le 1}$, and $U'\subset V$ is a neighborhood of zero.

We may take $U$ to be closed and convex and balanced (i.e., $zU\subset U$ for $|z|\le 1$). 
Since every $\mu \in M(K)_{\le 1}$ can be written as $\mu_1 - \mu_2 +i\mu_3-i\mu_4$, where the $\mu_i$'s are positive measures, also in $M(K)_{\le 1}$, it's enough to show that if $\mu_1$ is a positive measure, there are neighborhoods as above such that when $\mu_2 \in \mu_1+M$ and $v\in U'$, we have $\mu_1(v)-\mu_2(v)\in U$. 

By continuity of the action map $G\times V\to V$, there is a finite open cover $(K_i)_{i=1}^r$ of $K$, and an open $U'\subset V$, such that, for each $i$, 
$$ \{ v_1- v_2| v_1, v_2 \in \overline{\text{co}}(\pi(K_i)(U'))\} \subset \frac{1}{2} U.$$

There is a partition of unity $1_K=\sum_i f_i$, subordinate to the cover $(K_i)_i$, where the $f_i$'s are continuous functions and $0\le f_i\le 1$. 
Let $\mu_1, \mu_2$ be positive measures on $K$, with $\Vert \mu_i\Vert\le 1$, whose difference is contained in the weak-* neighborhood determined by the conditions
$$ |\left< \mu_1-\mu_2, f_i \right>| <\epsilon,$$
for given $\epsilon >0$ (to be determined). Let $m_i^j = \int_K f_i \mu_j$ for $j=1,2$.
Then, for $v\in U'$
$$ \pi(\mu_1)(v) - \pi(\mu_2) (v) = \sum_i \int_K f_i(g) \pi(g)(v) (\mu_1-\mu_2)(g).$$
Since the $\mu_i$'s and the $f_i$'s are positive, the integral 
$$ \int_K f_i(g) \pi(g)(v) \mu_j(g)$$
lies in $m_i^j \cdot \overline{\text{co}}(\pi(K_i)(U'))$. Thus, 
$$\int_K f_i(g) \pi(g)(v) (\mu_1-\mu_2)(g) \in \overline{\text{co}}\left(m_i^1 \pi(K_i)(U')\right) - \overline{\text{co}}\left(m_i^2 \pi(K_i)(U')\right)$$ 
$$\subset  m_i^1 \left[\overline{\text{co}}\left(\pi(K_i)(U')\right) - \overline{\text{co}}\left(\pi(K_i)(U')\right) \right] + \epsilon \overline{\text{co}}\left(\pi(K_i)(U')\right)  \subset \frac{m_i^1}{2} \cdot U + \epsilon \overline{\text{co}}\left(\pi(K_i)(U')\right),$$
and therefore 
$$ \pi(\mu_1)(v) - \pi(\mu_2) (v) \in \sum_i (\frac{m_i^1}{2} \cdot U) + \sum_i \epsilon \cdot \overline{\text{co}}\left(\pi(K_i)(U')\right)$$
$$\subset \frac{\Vert \mu_1\Vert}{2} \cdot U + \sum_i \epsilon \cdot \overline{\text{co}}\left(\pi(K_i)(U')\right) \subset \frac{1}{2} U + \sum_i \epsilon \cdot \overline{\text{co}}\left(\pi(K_i)(U')\right).$$
Choosing $\epsilon$ small enough, we can guarantee that this belongs to $U$, and we are done. 

\end{proof}






\section{Matrix coefficients} 
\label{section-matrixcoefficients}

For the moment we are working with an arbitrary topological coefficient field $k$.

The dual of a topological representation is defined as follows:
\begin{definition}
\label{definition-dual-representation}
Given a representation $(\pi, V)$ of a topological group on a locally convex topological vector space $V$, the dual representation $(\pi^*,V^*)$ on the dual space of continuous functionals on $V$ is defined by the property
 $$ \left< \pi^*(g)(v^*), v\right> = \left< v^*, \pi(g^{-1}) v\right>.$$
\end{definition}

Here, $V^*$ must be endowed with a suitable topology making the action continuous. For example, if $G$ is locally compact, then one can consider the topology of uniform convergence on compact sets:

\begin{lemma}
\label{lemma-topology-dual}
 Let $(\pi, V)$ be a representation of a localy compact group $G$. Then, the dual map $G\times V^*\to V^*$ is continuous when $V^*$ is endowed with the topology of uniform convergence on compact subsets of $V$. If $V$ is a Banach space, this topology is the finest topology which coincides with the weak-* topology on every norm-bounded subset of $V^*$.
\end{lemma}

\begin{proof}
 Given a compact $U\subset V$, and a compact $K\subset G$, the set $\pi(K)(U)$ is compact. Moreover, 
if we fix $U$ and a neighborhood $V_1$ of zero in $V$, we can find a compact neighborhood $K$ of $1\in G$ such that $U_K\subset V_1$, where $U_K$ is the set
 $$U_K=\{\pi(g)(v)-v| g\in K, v\in U\}\subset V.$$
(These facts follow directly from the continuity of the action of $G$ on $V$.)
 
 Fix $U$, $\epsilon>0$, and a vector $w\in V^*$, and let $V_1 = \{ v\in V| \left< w, v\right> <\epsilon\}$. Choose a compact neighborhood $K$ of $1\in G$ such that $U_K\subset V_1$. Let $W\subset V^*$ be the neighborhood of $w$, in the compact-open topology, of all vectors $w'$ with
 $$ |\left< w-w', \pi(K)U\right>|<\epsilon.$$
 
 Then, for all $g\in K$ and $w'\in W$, we have 
 $$ | \left< \pi^*(g^{-1}) (w')-w, U\right> | \le |\left< \pi^*(g^{-1}) (w'-w), U\right> | + | \left< \pi^*(g^{-1}) (w)-w, U\right> | =$$
 $$ = |\left< (w'-w), \pi(g) U\right> | + | \left< w, (\pi(g) - 1) U\right> | < \epsilon + \epsilon.$$

This shows continuity in the compact-open topology on $V^*$.

In the case of normed spaces, this topology coincides with the \emph{bounded weak-* topology}, the finest topology which coincides with the weak-* topology on every norm-bounded subset of $V^*$ \cite[\S II.5, Lemma 2]{Day}. (See also \cite[\S 2.7 and 3.4]{Megginson}.)
\end{proof}



When $V$ is a Banach space, it is not true, in general, that the action is continuous with respect to the norm topology on $V^*$ --- just consider $V^*=M(G)$ as the dual of $V=C(G)$, for a compact group $G$. However, for locally compact groups and \emph{reflexive} Banach spaces ($V^{**}=V$), this is the case, as we will see in Proposition \ref{proposition-reflexive-dual}.



\begin{remark}
\label{remark-spaces-induality}
 For two spaces $X$ and $Y$, not necessarily linear, which are in some sort of duality, in the sense that they come with a map 
 $$\left< \,\, , \,\, \right> : X\times Y\to Z,$$ 
 where $Z$ is another space, a \emph{right} action of a group $G$ on $X$ naturally induces a \emph{left} action on $Y$ (when $G$ is assumed to act trivially on the target $Z$), and vice versa:
 $$  \left< x\cdot g, y \right> = \left< x, g\cdot y \right>.$$
 Therefore, we need to replace $g$ by $g^{-1}$ on $X$, if we want left actions on both spaces.
\end{remark}

The remark above is exemplified in the following basic example:

\begin{definition}
 \label{definition-regular-representation}
If $k$ is the field of coefficients, and $F(H)$ denotes the space of $k$-valued functions on a group $H$, the left ($L$) and right ($R$) {\it regular representations} of $H$ on $F(H)$ are defined by 
$$ L \times R (h_1, h_2) (f) (x) = f(h_1^{-1} x h_2).$$
The term applies to any $H\times H$-invariant subspace of $F(H)$, and various $H\times H$-invariant quotients thereof (e.g., $L^p(H)$). 
\end{definition}

The right and left regular representations are a special case of Example \ref{example-representation} for $X=H$ and $G=H\times H$ (with left multiplication defined as a right action, $(g,x)\mapsto g^{-1}x$).

\begin{definition}
\label{definition-matrix-coefficient}
Given a representation $(\pi, V)$ of a topological group $G$, the {\it matrix coefficient} map is the $G\times G$-equivariant map 
$$M_\pi: \pi^* \otimes \pi \to F(G)$$
given by 
$$ v^*\otimes v \mapsto \left(g \mapsto \left< v^*, \pi(g) v\right> \right).$$
\end{definition}

\begin{lemma}
\label{lemma-matrixcoefficient-continuous}
The image of the matrix coefficient map (Definition \ref{definition-matrix-coefficient}) lies in the space $C(G)$ of continuous functions on $G$, and the resulting map 
$$ V^* \times V \to C(G)$$
is continuous when $C(G)$ is endowed with the seminorms of local uniform convergence.
\end{lemma}

\begin{proof}
 This follows immediately from the continuity of the map $G\times V\to V$.
\end{proof}


We return to the case $k=\mathbb C$, and $G$ a locally compact topological group. In that case, the spaces $M_c(G)$ and $C(G)$ are in duality, that is, there is a natural continuous map
$$ M_c(G) \otimes C(G) \to \mathbb C.$$

\begin{lemma}
 \label{lemma-mc-dualto-operator}
Given a topological representation $(\pi, V)$ of  a locally compact topological group $G$ on a space satisfying the conditions of \ref{proposition-integral-lcs} (e.g., a Fr\'echet space), we have, for every $\mu\in M_c(G)$, $v\in V$, $v^*\in V^*$,
\begin{equation}
 \label{equation-mc-dualto-operator}
\left< \mu , M_\pi (v^*\otimes v)\right> = \left< v^*, \pi(\mu) v \right>,
\end{equation}
where the first pairing is between measures and continuous functions, while the second is between $V$ and its dual.
\end{lemma}

\begin{proof}
 Just an unfolding of the definitions:
$$\left< \mu , M_\pi (v^*\otimes v)\right> =  \int_G \left< v^*, \pi(g) v\right>  \mu(g) = \left< v^*, \int_G \pi(g) v \mu(g) \right> = \left< v^* , \pi(\mu) v\right>.$$
\end{proof}







 
 
\section{Banach representations of compactly generated groups}
\label{section-Banach-representations}

\begin{definition}
\label{definition-radial-function}
 A {\it radial function} $r: G\to \mathbb R_+$, in the language of 
 \cite{Bernstein-Plancherel}
is a locally bounded positive function on $G$ such that $r(g_1\cdot g_2) \le r(g_1) + r(g_2)$. 

Two radial functions $r, r'$ are said to be {\it equivalent} if $(r+1)$ is comparable to $(r'+1)$, that is, there is a constant $C>0$ such that $C^{-1} (r+1)\le (r'+1) \le C (r+1)$.
\end{definition}

Suppose that $G$ is compactly generated and locally compact. Then there is a canonical equivalence class of radial functions on $G$, a representative of which is described by taking a compact generating subset $K$ with non-empty interior, and setting $r(g) = \min\{k| g\in K^k\}$. We will be working with such radial functions unless otherwise stated, and calling them ``natural''. 


\begin{lemma}
\label{lemma-bounded-by-radial}
Let $G$ be a compactly generated group, and $(\pi,V)$ a Banach representation of $G$.
There is a constant $C\ge 1$, depending on $V$ and the choice of radial function $r$, such that $\Vert \pi(g)\Vert \le C^{r(g)}$.
\end{lemma}

\begin{proof}
 From the definitions, if $K$ is a compact generating subset, defining the scale function $r$, then there is a constant $C$ such that $\Vert\pi(g)\Vert \le C$ for every $g\in K$, and therefore $\Vert\pi(g)\Vert \le C^{r(g)}$ for every $g\in G$.
\end{proof}

\begin{proposition}
\label{proposition-integral-Banach}
Let $G$ be a compactly generated, locally compact group, and $(\pi, V)$ a Banach representation of $G$. Endow $\text{End}(V)$ with the operator norm. The map: $M_c(G)\to \text{End}(V)$ is continuous, and bounded by the norm $\mu\mapsto \Vert \mu \cdot C^r\Vert$ on $M_c(G)$, for some natural radial function $r$ and constant $C \ge 1$.
\end{proposition}

\begin{proof}
 By \cite[Theorem 3.29]{Rudin}, 
 $$
 \left\Vert \int_G \pi(g)(v) \mu(g)\right\Vert \le \int_G \Vert \pi(g)\Vert |\mu|(g),
 $$
 and by Lemma \ref{lemma-bounded-by-radial} this is $\le \int_G C^{r(g)} |\mu|(g)$. 
\end{proof}


\begin{remark}
\label{remark-continuity-algebras}
Returning to Remark \ref{remark-continuity-operatortopology},  this proposition shows why it is not natural to require from the map $G\to \text{Aut}(V)\hookrightarrow \text{End}(V)$ to be continuous in the norm topology for $\text{End}(V)$: We can identify the action of elements $g\in G$ with the action of the corresponding delta measures, but in the space of measures we do not have $\delta_{g_n}\to \delta_g$ when $g_n\to g$.
\end{remark}


\begin{definition}
\label{definition-rapidly-decaying}
 Let $G$ be a compactly generated, locally compact group. The algebra $M_{rd}(G)$ of {\it rapidly decaying} measures on $G$ is the Fr\'echet subalgebra of $M(G)$ defined by the norms $\Vert \mu \cdot C^r\Vert$, for a natural radial function $r$ and all $C \ge 1$. 
\end{definition}

\begin{proposition}
\label{proposition-rapiddecay-Banach}
Every Banach representation extends to a continuous homomorphism $M_{rd}(G)\to \text{End}(V)$. 
\end{proposition}

\begin{proof}
 Follows immediately from Proposition \ref{proposition-integral-Banach}.
\end{proof}

Finally, a result that was mentioned earlier, which shows that the dual of a representation on a reflexive Banach space is continuous in the norm topology:

\begin{proposition}
 \label{proposition-reflexive-dual}
Let $(\pi,V)$ be a representation of a locally compact group on a reflexive Banach space. Then the dual representation $(\pi^*, V^*)$ is continuous in the norm topology. 
\end{proposition}

\begin{proof}
Let $W\subset V^*$ denote the subspace of elements $w$ for which the map 
$$ G \ni g \mapsto \pi^*(g) (w)\in V^*$$
is continuous in the norm topology for $V^*$. It will suffice to prove that $W=V^*$. Indeed, we also know (as a special case of Lemma \ref{lemma-bounded-by-radial}) that the operators $\pi(g)$, and hence their adjoints $\pi^*(g)$, are uniformly bounded for $g$ in any compact set. Thus, for every $g$ in a fixed compact neighborhood of the identity, and any $\epsilon >0$, there is a $\delta>0$ such that 
$$ \Vert w' - w\Vert <\delta \Rightarrow \Vert \pi^*(g)(w') - \pi^*(g)(w)\Vert <\epsilon.$$
Hence, if we can guarantee that $\Vert \pi^*(g)(w) - w\Vert$ is sufficiently small, for $w\in W$ and $g$ sufficiently close to $1$, the same holds for every $w'$ in a small neighborhood of $w$. 

The set $W$ is clearly a vector subspace, in particular convex. We will also show below that it is weakly-* dense. Since $V$ is reflexive, this is the same as weakly dense. \emph{The weak and strong closures of convex sets coincide} (as a corollary of the Hahn--Banach theorem), and therefore we will conclude that $W=V$.

There remains to show that $W$ is weakly-* dense. Fix a left Haar measure $dg$, and consider the map $C_c^\infty(G)\otimes V^*\to V^*$ given by 
\begin{equation}
\label{equation-C-to-Vstar} 
f \otimes v^*\mapsto \pi^*(fdg)(v^*).
\end{equation}
If we fix $v^*$ and a compact $K\subset G$, this becomes a continuous map
$$ C(K)\to V^*$$
with respect to the norm topology on $V^*$; indeed, we have
$$ \Vert \pi^*(fdg)(v^*) \Vert = \sup_{\Vert v\Vert = 1} |\left< v, \pi^*(fdg)(v^*)\right> | = $$
$$ = \sup_{\Vert v\Vert = 1} |\left< \pi(f^*dg) (v), (v^*)\right> | \ll_K \Vert f\Vert \cdot \Vert v^*\Vert,$$
by an application of Proposition \ref{proposition-integral-Banach}, where we have set $f^*(g) = f(g^{-1})$, and the symbol $\ll_K a$ means $< c\cdot a$, for a constant $c$ depending on $K$. But the map \eqref{equation-C-to-Vstar}, for a fixed $v^*$, is equivariant with respect to the left regular action of $G$ on $C_c(G)$, which is continuous. Therefore, the image of this map belongs to $W$.


Moreover, $v^*$ is in the weak-* closure of this map: it is enough to consider a sequence of positive continuous functions $f_n$ with $\int f_n dg=1$ and $f_n$ converging weakly to the delta measure at the identity, and apply Proposition \ref{proposition-integral-lcs} to the representation $\pi$ to deduce that $\pi^*(f_ndg)(v^*)\to v^*$ in the weak-* topology. Thus, we have shown that $W$ is weakly-* dense, completing the proof.


 
 
 
\end{proof}


\begin{remark}
\label{remark-dualcontinuous-withoutcompactness}
 Proposition  \ref{proposition-reflexive-dual}
 holds even without the assumption on local compactness of $G$, see \cite[Corollary 6.9]{Megrelishvili}.
\end{remark}

If the Banach space $V$ is not reflexive, then $V^*$ may fail to be a continuous representation with the norm topology. In this case, we can work with the subspace of $V^*$ for which this is true:

\begin{definition}
 \label{definition-contragredient}
The {\it contragredient} of a Banach representation $(\pi, V)$ of a topological group $G$ is the natural representation $(\tilde\pi, \tilde V)$ of $G$ on the subspace $\tilde V\subset V^*$ consisting of those vectors $v^*$ such that the orbit map $G\ni g\mapsto g\cdot v^* \in V^*$ is (norm) continuous. 
\end{definition}


\begin{lemma}
 \label{lemma-contragredient}
The subspace $\tilde V\subset V^*$ of the contragredient representation is closed. 
\end{lemma}

\begin{proof}
 See \cite[\S 3.1]{Bernstein-Kroetz} for references. 
\end{proof}

\begin{example}
 \label{example-contragredient}
If $G$ is a compact group, and $V=L^1(G)$ under the regular representation, then $\tilde V = C(G)$, but $V^*=L^\infty(G)$. Moreover, $\tilde{\tilde V} = V$, while $(\tilde V)^* = M(G)$, the space of Radon measures.
\end{example}




\section{Unitary representations}
\label{section-unitary-representations}

We continue assuming that the coefficient field is $\mathbb C$, and that the topological group $G$ is locally compact.


\begin{definition}
\label{definition-unitary-representation}
A {\it unitary} representation of $G$ is a representation of $G$ on a Hilbert space\footnote{We will assume throughout that Hilbert spaces are separable.} $V$ (over $\mathbb C$) which preserves the norm (i.e.\ $\pi$ has image in the subgroup of unitary transformations, $U(V)\subset \text{Aut}(V)$). 

A representation $(\pi,V)$ of $G$ on a topological vector space $V$ is {\it unitarizable} if $V$ admits a (continuous, positive definite) inner product such that the corresponding Hilbert space completion is unitary.
\end{definition}

If $V$ is a Hilbert space, the algebra $B(V)$ of bounded operators on $V$ is a {\it $C^*$-algebra}: a Banach algebra, with an involution $T\mapsto T^*$ (the adjoint of an operator), and the property that $\Vert T^*T\Vert = \Vert T\Vert^2$.

$C^*$-algebras play an important role in the analysis of unitary representations of a group. 
The map $M_c(G)\to B(V)$ defines, by pullback, a seminorm on $M_c(G)$. Assume that $G$ is a locally compact group, with right Haar measure $dg$. The restriction of all those seminorms to $C_c(G)dg\subset M_c(G)$ defines a norm:
$$ \Vert fdg \Vert_{C^*} = \sup_{(\pi,V)} \Vert \pi(fdg)\Vert,$$
where $(\pi,V)$ ranges over all unitary representations of $G$. 

Considering just the right regular representation $R$ of $G$ on $L^2(G,dg)$, we obtain another norm
$$ \Vert fdg \Vert_{C_r^*} = \Vert R(fdg)\Vert.$$

\begin{definition}
\label{definition-Cstar}
The completion of $C_c(G) dg$ with respect to the norm $\Vert\bullet\Vert_{C^*}$ is the {\it $C^*$-algebra of $G$}. Its completion with respect to $\Vert\bullet \Vert_{C_r^*}$ is the {\it reduced $C^*$-algebra of $G$}.
\end{definition}

\section{Fr\'echet representations of moderate growth}
\label{section-Frepresentations}

We continue assuming that the coefficient field is $\mathbb C$. We also assume that the topological group $G$ is locally compact, and compactly generated. 

\begin{definition}
\label{definition-Frepresentation}
An {\it F-representation} of $G$ (in the language of 
\cite{Bernstein-Kroetz}), or {\it Fr\'echet representation of moderate growth}
is a countable (inverse) limit of Banach representations, that is, a representation on a Fr\'echet space $V$, such that $V$ admits an equivariant topological isomorphism 
$$ 
V = \lim_{\leftarrow} V_n,
$$
where the $V_n$'s are Banach representations of $G$.
\end{definition}
 
Note that this is {\it stronger} than just a Fr\'echet representation of $G$: In a Fr\'echet representation, for every (continuous) seminorm $p$, and for every compact $K\subset G$, there is a seminorm $q$ such that 
$$ p(\pi(g) (v)) \le q(v)$$
for $g\in K$, $v\in V$. For an F-representation, there is a complete system of seminorms $p_n$ such that we can take $q_n=c_{K,n}\cdot p$, where $c_{K,n}$ is a scalar that depends on $K$ and $n$.

Note that by Proposition \ref{proposition-rapiddecay-Banach}, the action of $M_c(G)$ on $V$ extends to the measures $M_{rd}(G)$ of rapid decay. 

Fix a (natural) radial function $r$, and let $\Vert g\Vert:= e^{r(g)}$. The following definition is due to \cite{Casselman-canonicalextensions}:

\begin{definition}
\label{definition-moderate-growth}
A Fr\'echet representation $(\pi,V)$ of $G$ is said to be of {\it moderate
growth} if for any (continuous) seminorm $p$ on $V$ there exists a seminorm $q$
and an integer $N > 0$ such that
\begin{equation}
\label{equation-moderate-growth}
p(\pi(g)v) \le \Vert g\Vert^N q(v)
\end{equation}
for all $g \in G$.
\end{definition}

\begin{lemma}
\label{lemma-F-moderate-growth} 
 Let $(\pi, V)$ be a Fr\'echet representation of the Lie group
$G$. Then the following statements are equivalent:
\begin{enumerate}
 \item  $(\pi,V)$ is of moderate growth;
 \item $(\pi, V)$ is an F-representation.
\end{enumerate}
\end{lemma}

\begin{proof}
We follow \cite[Lemma 2.10]{Bernstein-Kroetz}.

If $(\pi, V)$ if of moderate growth, $p$ is any seminorm, and $q$, $N$ are a seminorm and a positive number satisfying \eqref{equation-moderate-growth}, setting 
$$ \tilde p (v) = \sup_{g\in G} \frac{p(\pi(g)v)}{\Vert g\Vert^N}$$ 
we have inequalities $p\le \tilde p\le q$, and $\tilde p(\pi(g)) \le \Vert g\Vert^N \tilde p(v)$. The former implies that the seminorms of the form $\tilde p$ define the topology, and the latter implies that they are $G$-continuous.

Vice versa, if $(\pi, V)$ is an F-representation, it suffices to prove the moderate growth condition for a system of $G$-continuous seminorms defining the topology. But then it reduces to the case of Banach representations, Lemma \ref{lemma-bounded-by-radial}. 
\end{proof}

\begin{remark}
\label{remark-generalscalefunctions}
 Bernstein and Kr\"otz introduce a more general notion of F-representations, which allows for more general scale functions on the group than the one defined in \ref{section-Banach-representations}.
\end{remark}


\section{The $C^*$-algebra of a group, and Plancherel decomposition of unitary representations}
\label{section-Plancherel}

In this section, we collect results from the theory of $C^*$-algebras (which, in turn, rely on results on the theory of Von Neumann-, or $W^*$-, algebras. We point the reader to \cite{Dixmier-Cstar} for complete definitions and proofs.





%**************************************************************************************


%*************************************************************************





\begin{multicols}{2}[\section{Other chapters}]
\noindent
\begin{enumerate}
\item \hyperref[introduction-section-phantom]{Introduction}
\item \hyperref[representationtheory-section-phantom]{Basic Representation Theory}
\item \hyperref[representations-compact-section-phantom]{Representations of compact groups}
\item \hyperref[liegroups-general-section-phantom]{Lie groups and Lie algebras: general properties}
\item \hyperref[liestructure-section-phantom]{Structure of finite-dimensional Lie algebras}
\item \hyperref[algebraicgroups-section-phantom]{Linear algebraic groups}
\item \hyperref[reductiveforms-section-phantom]{Forms and covers of reductive groups, and the $L$-group}
\item \hyperref[vermamodules-section-phantom]{Verma modules}
\item \hyperref[representations-local-section-phantom]{Representations of reductive groups over local fields}
%\item \hyperref[gKmodules-section-phantom]{$(\mathfrak g, K)$-modules}
%\item \hyperref[asymptotics-section-phantom]{Asymptotics and the Langlands classification}
\item \hyperref[plancherel-section-phantom]{Plancherel formula: reduction to discrete spectra}
\item \hyperref[discreteseries-section-phantom]{Construction of discrete series}
\item \hyperref[galoiscohomology-section-phantom]{Galois cohomology of linear algebraic groups}
\item \hyperref[automorphicspace-section-phantom]{The automorphic space}
%\item \hyperref[harmonicanalysis-section-phantom]{Harmonic analysis over local fields}
%\item \hyperref[automorphicforms-section-phantom]{Automorphic forms}
%\item \hyperref[periods-section-phantom]{Periods, theta correspondence, related methods}
%\item \hyperref[traceformulalocal-section-phantom]{The trace formula: local aspects}
%\item \hyperref[traceformulaglobal-section-phantom]{The trace formula: global aspects}
%\item \hyperref[arithmetic-section-phantom]{Arithmetic, reciprocity, Shimura varieties}
%\item \hyperref[geometric-section-phantom]{Geometric aspects}
\item \hyperref[fdl-section-phantom]{GNU Free Documentation License}
\item \hyperref[index-section-phantom]{Auto Generated Index}
\end{enumerate}
\end{multicols}





\bibliography{my}
\bibliographystyle{amsalpha}

\end{document}
